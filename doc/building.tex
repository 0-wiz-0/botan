\documentclass{article}

\setlength{\textwidth}{6.5in}
\setlength{\textheight}{9in}

\setlength{\headheight}{0in}
\setlength{\topmargin}{0in}
\setlength{\headsep}{0in}

\setlength{\oddsidemargin}{0in}
\setlength{\evensidemargin}{0in}

\title{\textbf{Botan Build Guide}}
\author{Jack Lloyd \\
        \texttt{lloyd@randombit.net}}
\date{2009-10-09}

\newcommand{\filename}[1]{\texttt{#1}}
\newcommand{\module}[1]{\texttt{#1}}

\newcommand{\type}[1]{\texttt{#1}}
\newcommand{\function}[1]{\textbf{#1}}
\newcommand{\macro}[1]{\texttt{#1}}

\begin{document}

\maketitle

\tableofcontents

\parskip=5pt
\pagebreak

\section{Introduction}

This document describes how to build Botan on Unix/POSIX and MS
Windows systems. The POSIX oriented descriptions should apply to most
common Unix systems (including MacOS X), along with POSIX-ish systems
like BeOS, QNX, and Plan 9. Currently, systems other than Windows and
POSIX (such as VMS, MacOS 9, OS/390, OS/400, ...) are not supported by
the build system, primarily due to lack of access. Please contact the
maintainer if you would like to build Botan on such a system.

Botan's build is controlled by configure.py, which is a Python
script. Python 2.4 or later is required.

\section{For the Impatient}

\begin{verbatim}
$ ./configure.py [--prefix=/some/directory]
$ make
$ make install
\end{verbatim}

Or using \verb|nmake|, if you're compiling on Windows with Visual
C++. On platforms that do not understand the '\#!' convention for
beginning script files, or that have Python installed in an unusual
spot, you might need to prefix the \texttt{configure.py} command with
\texttt{python} or \texttt{/path/to/python}.

\section{Building the Library}

The first step is to run \filename{configure.py}, which is a Python
script that creates various directories, config files, and a Makefile
for building everything. The script requires at least Python 2.4; any
later version of Python 2.x should also work.

The script will attempt to guess what kind of system you are trying
to compile for (and will print messages telling you what it guessed).
You can override this process by passing the options \verb|--cc|,
\verb|--os|, and \verb|--cpu|.

You can pass basically anything reasonable with \verb|--cpu|: the
script knows about a large number of different architectures, their
sub-models, and common aliases for them. You should only select the
64-bit version of a CPU (such as ``sparc64'' or ``mips64'') if your
operating system knows how to handle 64-bit object code -- a 32-bit
kernel on a 64-bit CPU will generally not like 64-bit code.

By default the script tries to figure out what will work on your
system, and use that. It will print a display at the end showing which
algorithms have and have not been enabled. For instance on one system
we might see lines like:

\begin{verbatim}
 INFO: Skipping mod because CPU incompatible - asm_amd64 mp_amd64 mp_asm64 sha1_amd64
 INFO: Skipping mod because OS incompatible - cryptoapi_rng win32_stats
 INFO: Skipping mod because compiler incompatible - mp_ia32_msvc
 INFO: Skipping mod because loaded on request only - bzip2 gnump openssl qt_mutex zlib
\end{verbatim}

The ones that are 'loaded on request only' have to be explicitly asked
for, because they rely on third party libraries which your system
might not have. For instance to enable zlib support, add
\verb|--with-zlib| to your invocation of \verb|configure.py|.

You can control which algorithms and modules are built using the
options ``\verb|--enable-modules=MODS|'' and
``\verb|--disable-modules=MODS|'', for instance \\
``\verb|--enable-modules=blowfish,md5,rsa,zlib --disable-modules=arc4,cmac|''.
Modules not listed on the command line will simply be loaded if needed
or if configured to load by default.

The script tries to guess what kind of makefile to generate, and it
almost always guesses correctly (basically, Visual C++ uses NMAKE with
Windows commands, and everything else uses Unix make with POSIX
commands). Just in case, you can override it with
\verb|--make-style=somestyle|. The styles Botan currently knows about
are 'unix' (normal Unix makefiles), and 'nmake', the make variant
commonly used by Windows compilers. To add a new variant (eg, a build
script for VMS), you will need to create a new template file in
\filename{src/build-data/makefile}.

\subsection{POSIX / Unix}

The basic build procedure on Unix and Unix-like systems is:

\begin{verbatim}
   $ ./configure.py [--enable-modules=<list>] [--cc=CC]
   $ make
   # You may need to set your LD_LIBRARY_PATH or equivalent for ./check to run
   $ make check # optional, but a good idea
   $ make install
\end{verbatim}

This will probably default to using GCC, depending on what can be
found within your PATH.

The \verb|make install| target has a default directory in which it
will install Botan (typically \verb|/usr/local|). You can override
this by using the \texttt{--prefix} argument to
\filename{configure.py}, like so:

\verb|./configure.py --prefix=/opt <other arguments>|

On some systems shared libraries might not be immediately visible to
the runtime linker. For example, on Linux you may have to edit
\filename{/etc/ld.so.conf} and run \texttt{ldconfig} (as root) in
order for new shared libraries to be picked up by the linker. An
alternative is to set your \texttt{LD\_LIBRARY\_PATH} shell variable
to include the directory that the Botan libraries were installed into.

\subsection{MS Windows}

The situation is not much different here. We'll assume you're using Visual C++
(for Cygwin, the Unix instructions are probably more relevant). You need to
have a copy of Python installed, and have both Python and Visual C++ in your path.

\begin{verbatim}
   > python configure.py --cc=msvc (or --cc=gcc for MinGW) [--cpu=CPU]
   > nmake
   > nmake check # optional, but recommended
\end{verbatim}

For Win95 pre OSR2, the \verb|cryptoapi_rng| module will not work,
because CryptoAPI didn't exist. And all versions of NT4 lack the
ToolHelp32 interface, which is how \verb|win32_stats| does its slow
polls, so a version of the library built with that module will not
load under NT4. Later systems (98/ME/2000/XP) support both methods, so
this shouldn't be much of an issue.

Unfortunately, there currently isn't an install script usable on
Windows. Basically all you have to do is copy the newly created
\filename{libbotan.lib} to someplace where you can find it later (say,
\verb|C:\botan\|). Then copy the entire \verb|build\include\botan|
directory, which was constructed when you built the library, into the
same directory.

When building your applications, all you have to do is tell the
compiler to look for both include files and library files in
\verb|C:\botan|, and it will find both. Or you can move them to a
place where they will be in the default compiler search paths (consult
your documentation and/or local expert for details).

\pagebreak

\subsection{Configuration Parameters}

There are some configuration parameters which you may want to tweak
before building the library. These can be found in
\filename{config.h}. This file is overwritten every time the configure
script is run (and does not exist until after you run the script for
the first time).

Also included in \filename{build/build.h} are macros which are defined
if one or more extensions are available. All of them begin with
\verb|BOTAN_HAS_|. For example, if \verb|BOTAN_HAS_COMPRESSOR_BZIP2|
is defined, then an application using Botan can include
\filename{<botan/bzip2.h>} and use the Bzip2 filters.

\macro{BOTAN\_MP\_WORD\_BITS}: This macro controls the size of the
words used for calculations with the MPI implementation in Botan. You
can choose 8, 16, 32, or 64, with 32 being the default. You can use 8,
16, or 32 bit words on any CPU, but the value should be set to the
same size as the CPU's registers for best performance. You can only
use 64-bit words if an assembly module (such as \module{mp\_ia32} or
\module{mp\_asm64}) is used. If the appropriate module is available,
64 bits are used, otherwise this is set to 32. Unless you are building
for a 8 or 16-bit CPU, this isn't worth messing with.

\macro{BOTAN\_VECTOR\_OVER\_ALLOCATE}: The memory container
\type{SecureVector} will over-allocate requests by this amount (in
elements). In several areas of the library, we grow a vector fairly often. By
over-allocating by a small amount, we don't have to do allocations as often
(which is good, because the allocators can be quite slow). If you \emph{really}
want to reduce memory usage, set it to 0. Otherwise, the default should be
perfectly fine.

\macro{BOTAN\_DEFAULT\_BUFFER\_SIZE}: This constant is used as the size of
buffers throughout Botan. A good rule of thumb would be to use the page size of
your machine. The default should be fine for most, if not all, purposes.

\macro{BOTAN\_GZIP\_OS\_CODE}: The OS code is included in the Gzip header when
compressing. The default is 255, which means 'Unknown'. You can look in RFC
1952 for the full list; the most common are Windows (0) and Unix (3). There is
also a Macintosh (7), but it probably makes more sense to use the Unix code on
OS X.

\subsection{Multiple Builds}

It may be useful to run multiple builds with different
configurations. Specify \verb|--build-dir=<dir>| to set up a build
environment in a different directory.

\subsection{Local Configuration}

You may want to do something peculiar with the configuration; to
support this there is a flag to \filename{configure.py} called
\texttt{--with-local-config=<file>}. The contents of the file are
inserted into \filename{build/build.h} which is (indirectly) included
into every Botan header and source file.

\pagebreak

\section{Modules}

There are a fairly large number of modules included with Botan. Some
of these are extremely useful, while others are only necessary in very
unusual circumstances. The modules included with this release are:

\newcommand{\mod}[2]{\textbf{#1}: #2}

\begin{list}{$\cdot$}
  \item \mod{alloc\_mmap}{Allocates memory using memory mappings of temporary
         files. This means that if the OS swaps all or part of the application,
         the sensitive data will be swapped to where we can later clean it,
         rather than somewhere in the swap partition.}

  \item \mod{bzip2}{Enables an application to perform bzip2 compression
         and decompression using the library. Available on any system that has
         bzip2.}

  \item \mod{zlib}{Enables an application to perform zlib compression and
         decompression using the library. Available on any system that has
         zlib.}

  %\item \mod{eng\_aep}{An engine that uses any available AEP accelerator card
  %       to speed up PK operations. You have to have the AEP drivers installed
  %       for this to link correctly, but you don't have to have a card
  %       installed - it will automatically be enabled if a card is detected at
  %       run time.}

  \item \mod{gnump}{An engine that uses GNU MP to speed up PK operations.
         GNU MP 4.1 or later is required.}

  \item \mod{openssl}{An engine that uses OpenSSL to speed up public key
                      operations and some ciphers/hashes. OpenSSL 0.9.7 or
                      later is required.}

  \item \mod{beos\_stats}{An entropy source that uses BeOS-specific
    APIs to gather (hopefully unpredictable) data from the system.}

  \item \mod{cryptoapi\_rng}{An entropy source that uses the Win32
    CryptoAPI function \texttt{CryptGenRandom} to gather
    entropy. Supported on NT4, Win95 OSR2, and all later Windows
    systems.}

  \item \mod{egd}{An entropy source that accesses EGD (the entropy
         gathering daemon). Common on Unix systems that don't have
         \texttt{/dev/random}.}

  \item \mod{proc\_walk}{Gather entropy by reading files from a particular file
          tree. Usually used with \texttt{/proc}; most other file trees don't
          have sufficient variability over time to be useful.}

  \item \mod{unix\_procs}{Gather entropy by running various Unix programs, like
          \texttt{arp} and \texttt{vmstat}, and reading their output in the
          hopes that at least some of it will be unpredictable to an attacker.}

  \item \mod{win32\_stats}{Gather entropy by walking through various pieces of
          information about processes running on the system. Does not run on
          NT4, but should run on all other Win32 systems.}

  \item \mod{fd\_unix}{Let the users of \texttt{Pipe} perform I/O with Unix
         file descriptors in addition to \texttt{iostream} objects.}

  \item \mod{pthread}{Add support for using \texttt{pthread} mutexes to
         lock internal data structures. Important if you are using threads
         with the library.}

  \item \mod{qt\_mutex}{Add support for using Qt mutexes to lock internal data
         structures.}

  \item \mod{cpu\_counter}{Use the contents of the CPU cycle counter when
         generating random bits to further randomize the results. Works on x86
         (Pentium and up), Alpha, and SPARCv9.}

  \item \mod{posix\_rt}{Use the POSIX realtime clock as a high-resolution
         timer.}

  \item \mod{gettimeofday}{Use the traditional Unix
    \texttt{gettimeofday} as a high resolution timer.}

  \item \mod{win32\_query\_perf\_ctr}{Use Win32's
    \texttt{QueryPerformanceCounter} as a high resolution timer.}

\end{list}

\pagebreak

\section{Building Applications}

\subsection{Unix}

Botan usually links in several different system libraries (such as
\texttt{librt} and \texttt{libz}), depending on which modules are
configured at compile time. In many environments, particularly ones
using static libraries, an application has to link against the same
libraries as Botan for the linking step to succeed. But how does it
figure out what libraries it \emph{is} linked against?

The answer is to ask the \filename{botan-config} script. This
basically solves the same problem all the other \filename{*-config}
scripts solve, and in basically the same manner.

There are 4 options:

\texttt{--prefix[=DIR]}: If no argument, print the prefix where Botan
is installed (such as \filename{/opt} or \filename{/usr/local}). If an
argument is specified, other options given with the same command will
execute as if Botan as actually installed at \filename{DIR} and not
where it really is; or at least where \filename{botan-config} thinks
it really is. I should mention that it

\texttt{--version}: Print the Botan version number.

\texttt{--cflags}: Print options that should be passed to the compiler
whenever a C++ file is compiled. Typically this is used for setting
include paths.

\texttt{--libs}: Print options for which libraries to link to (this includes
\texttt{-lbotan}).

Your \filename{Makefile} can run \filename{botan-config} and get the
options necessary for getting your application to compile and link,
regardless of whatever crazy libraries Botan might be linked against.

Botan also by default installs a file for \texttt{pkg-config},
namespaced by the major and minor versions. So it can be used,
for instance, as

\begin{verbatim}
$ pkg-config botan-1.8 --modversion
1.8.0
$ pkg-config botan-1.8 --cflags
-I/usr/local/include
$ pkg-config botan-1.8 --libs
-L/usr/local/lib -lbotan -lm -lbz2 -lpthread -lrt
\end{verbatim}

\subsection{MS Windows}

No special help exists for building applications on Windows. However,
given that typically Windows software is distributed as binaries, this
is less of a problem - only the developer needs to worry about it. As
long as they can remember where they installed Botan, they just have
to set the appropriate flags in their Makefile/project file.

\pagebreak

\section{Language Wrappers}

\subsection{Building the Python wrappers}

The Python wrappers for Botan use Boost.Python, so you must have Boost
installed. To build the wrappers, add the flag

\verb|--use-boost-python|

to \verb|configure.py|. This will create a second makefile,
\verb|Makefile.python|, with instructions for building the Python
module. After building the library, execute

\begin{verbatim}
$ make -f Makefile.python
\end{verbatim}

to build the module. Currently only Unix systems are supported, and
the Makefile assumes that the version of Python you want to build
against is the same one you used to run \verb|configure.py|.

To install the module, use the \verb|install| target.

Examples of using the Python module can be seen in \filename{doc/python}

\subsection{Building the Perl XS wrappers}

To build the Perl XS wrappers, change your directory to
\filename{src/wrap/perl-xs} and run \verb|perl Makefile.PL|, then run
\verb|make| to build the module and \verb|make test| to run the test
suite.

\begin{verbatim}
$ perl Makefile.PL
Checking if your kit is complete...
Looks good
Writing Makefile for Botan
$ make
cp Botan.pm blib/lib/Botan.pm
AutoSplitting blib/lib/Botan.pm (blib/lib/auto/Botan)
/usr/bin/perl5.8.8 /usr/lib64/perl5/5.8.8/ExtUtils/xsubpp  [...]
g++ -c   -Wno-write-strings -fexceptions  -g   [...]
Running Mkbootstrap for Botan ()
chmod 644 Botan.bs
rm -f blib/arch/auto/Botan/Botan.so
g++  -shared Botan.o  -o blib/arch/auto/Botan/Botan.so  \
           -lbotan -lbz2 -lpthread -lrt -lz     \

chmod 755 blib/arch/auto/Botan/Botan.so
cp Botan.bs blib/arch/auto/Botan/Botan.bs
chmod 644 blib/arch/auto/Botan/Botan.bs
Manifying blib/man3/Botan.3pm
$ make test
PERL_DL_NONLAZY=1 /usr/bin/perl5.8.8 [...]
t/base64......ok
t/filt........ok
t/hex.........ok
t/oid.........ok
t/pipe........ok
t/x509cert....ok
All tests successful.
Files=6, Tests=83,  0 wallclock secs ( 0.08 cusr +  0.02 csys =  0.10 CPU)
\end{verbatim}

\end{document}
