\documentclass{article}

\setlength{\textwidth}{6.5in}
\setlength{\textheight}{9in}

\setlength{\headheight}{0in}
\setlength{\topmargin}{0in}
\setlength{\headsep}{0in}

\setlength{\oddsidemargin}{0in}
\setlength{\evensidemargin}{0in}

\title{\textbf{Botan API Reference}}
\author{}
\date{2007/03/03}

\newcommand{\filename}[1]{\texttt{#1}}
\newcommand{\manpage}[2]{\texttt{#1}(#2)}

\newcommand{\macro}[1]{\texttt{#1}}

\newcommand{\function}[1]{\textbf{#1}}
\newcommand{\keyword}[1]{\texttt{#1}}
\newcommand{\type}[1]{\texttt{#1}}
\renewcommand{\arg}[1]{\textsl{#1}}
\newcommand{\namespace}[1]{\texttt{#1}}

\newcommand{\url}[1]{\texttt{#1}}

\newcommand{\ie}[0]{\emph{i.e.}}
\newcommand{\eg}[0]{\emph{e.g.}}

\begin{document}

\maketitle

\tableofcontents

\parskip=5pt

\pagebreak
\section{Introduction}

Botan is a C++ library which attempts to provide the most common cryptographic
algorithms and operations in an easy to use and portable package. Currently it
runs on a wide variety of systems, using numerous different compilers and on
many different CPU architectures.

The base library is written in ISO C++, so it can be ported with
minimal fuss, but Botan also supports a modules system. This system
exposes system dependent code to the library through portable
interfaces, extending the set of services available to users.

\subsection{Targets}

Botan's primary targets (system-wise) are 32 and 64-bit systems with
at least a few megabytes of memory. Generally, given the choice
between optimizing for 32-bit systems and 64-bit systems, Botan is
written to prefer 64-bit, simply on the theory that where performance
is a real concern, modern 64-bit processors are the obvious
choice. And also because two of the three machines owned by the
primary developer have 64-bit CPUs. But performance on 32 bit systems
is also quite good.

Today smaller systems, such as handhelds, set-top boxes, and the
bigger smart phones and smart cards, are also capable of using
Botan. However, Botan uses a fairly large amount of code space (up to
several megabytes, depending upon the compiler and options used),
which could be prohibitive in some systems. Usage of RAM is fairly
modest, usually under 64K.

Botan's design makes it quite easy to remove unused algorithms in such a way
that applications do not need to be recompiled to work, even applications that
use the algorithms in question. They can simply ask Botan if the algorithm
exists, and if Botan says yes, ask the library to give them such an object for
that algorithm.

\subsection{Why Botan?}

Botan may be the perfect choice for your application. Or it might be a
terribly bad idea. This section is basically to make it clear what Botan is
and is not.

First, let's cover the major strengths:

\begin{list}{$\cdot$}
  \item Support is (usually) quickly available on the project mailing lists.
        Commercial support licenses are available for those that desire them.

  \item
  \item Is written in a (fairly) clean object-oriented style, and the usual
        API works in terms of reasonably high-level abstractions.

  \item Supports a huge variety of algorithms, including most of the major
        public key algorithms and standards (such as IEEE 1363, PKCS, and
        X.509v3).

  \item Supports a name-based lookup scheme, so you can get ahold of any
        algorithm on the fly.

  \item You can easily extend much of the system at application compile time or
        at run time.

  \item Works well with a wide variety of compilers, operating systems, and
        CPUs, and more all the time.

  \item Is the only open source crypto library (that I know of) that has
        support for memory allocation techniques that prevent an attacker from
        reading swap in an attempt to gain access to keys or other secrets. In
        fact several different such methods are supported, depending on the
        system (two methods for Unix, another for Windows).

  \item Has (optional) support for Zlib and Bzip2 compression/decompression
        integrated completely into the system -- it only takes a line or two of
        code to add compression to your application.
\end{list}

\noindent
And the major downsides and deficiencies are:

\begin{list}{$\cdot$}
  \item It's written in C++. If your application isn't, Botan is probably
        going to be more pain than it's worth.
  \item

  \item Botan doesn't directly support higher-level protocols and
        formats like SSL or OpenPGP. SSH support is available from a
        third-party, and there is an alpha-level SSL/TLS library
        currently available.

  \item Doesn't support elliptic curve algorithms; ECDSA support is planned at
        some point, but demand seems quite low.

  \item Doesn't currently support any very high level 'envelope' style
        processing - support for this will probably be added once support for
        CMS is available, so code using the high level interface will produce
        data readable by many other libraries.
\end{list}

\pagebreak
\section{Getting Started}

\subsection{Basic Conventions}

With a very small number of exceptions, declarations in the library
are contained within the namespace \namespace{Botan}. Botan declares
several typedef'ed types to help buffer it against changes in machine
architecture.  These types are used extensively in the interface, and
thus it would be often be convenient to use them without the
\namespace{Botan} prefix. You can do so by \keyword{using} the
namespace \namespace{Botan\_types} (this way you can use the type
names without the namespace prefix, but the remainder of the library
stays out of the global namespace). The included types are \type{byte}
and \type{u32bit}, which are unsigned integer types.

The headers for Botan are usually available in the form
\filename{botan/headername.h}. For brevity in this documentation,
headers are always just called \filename{headername.h}, but they
should be used with the \filename{botan/} prefix in your actual code.

\subsection{Initializing the Library}

There are a set of core services which the library needs access to
while it is performing requests. To ensure these are set up, you must
create a \type{LibraryInitializer} object (using called 'init' in
Botan example code; 'botan\_library' or 'botan\_init' make more sense
in real code) prior to making any calls to Botan. This object's
lifetime must exceed that of all other Botan objects your application
creates; for this reason the best place to create the
\type{LibraryInitializer} is at the start of your \function{main}
function, since this guarantees that it will be created first and
destroyed last. The initializer does things like initializing the
memory allocation system, setting up the algorithm lookup tables,
finding out if there is a high resolution timer available to use, and
similar such matters. With no arguments, the library is initialized
with various default settings. So 99\% of the time, all you need is

\texttt{Botan::LibraryInitializer init;}

at the start of your \texttt{main}. If you're not doing anything
exotic, then you can safely skip the rest of this section.

The constructor takes an instance of another object, called
\type{InitializerOptions}, which specifies the settings of various
options. Normally you can ignore this and simply pass a human readable
string, which the \type{InitializerOptions} constructor will parse. An
empty string signifies using defaults; any options not specifically
mentioned in the initialization string also assume the compiled in
default.
If more than one option is used, they should be separated by a
space. Boolean arguments (all except for the ``config'' option) can
take an argument of ``true'' (or ``yes'') or ``false'' (or ``no'') to
explicitly turn them on or off. Simply giving the name of the option
without any argument signifies that the option should be toggled on.

\newcommand{\option}[1]{\noindent \textbf{Option ``#1''}}

\option{thread\_safe}: The library should use mutexes for guarding
access to shared resources, such as the memory allocation system. If you pass
the ``thread\_safe'' option, and the initializer can't find a useful mutex
module, it will throw an exception. Botan seems to work in threaded programs,
but it hasn't been tested thoroughly, and problems may remain. Note that Botan
is not thread safe at the object level; any objects shared between threads need
explicit locking.

\option{secure\_memory}: Try to create a more secure allocator type --
one that either locks allocated memory into RAM, or that memory maps a
disk file that it erases after use. If both are available, it will
prefer the memory mapping mechanism, because locking memory requires
privileges on many systems.

On systems that don't (currently) have any specialized allocators, like
MS Windows, this option is ignored.

\option{use\_engines}: Use any available ``engine'' modules to speed
up processing. Currently Botan has support for engines based on the
AEP1000/AEP2000 crypto hardware cards, GNU MP, and OpenSSL's BN
library. Further support for crypto acceleration hardware will be
added in future releases.

\option{fips140}: This option, in theory, toggles Botan into FIPS 140
mode. Please note that Botan \emph{has not} been FIPS 140 validated at
this time, and that a number of changes will be necessary before such
a validation could occur. Do not use this option.

\option{selftest}: Run some basic self tests during startup.
Specifically this runs a set of tests for DES, TripleDES, AES,
CMAC(AES), SHA-1, HMAC(SHA-1), SHA-256, and HMAC(SHA-256). This option
is enabled by default.

\option{seed\_rng}: Attempt to seed the global PRNGs at
startup. This option is toggled on by default, and can be disabled by passing
``seed\_rng=false''. This is primarily useful when you know that the built-in
library entropy sources will not work, and you are providing you own entropy
source(s) later on.

If you do not create a \type{LibraryInitializer} object, pretty much
any Botan operation will fail, because it will be unable to do basic
things like allocate memory or get random bits. Note too, that you
should be careful to only create one such object.

It is not strictly necessary to create a \type{LibraryInitializer};
the actual code performing the initialization and shutdown are in
static member functions of \type{LibraryInitializer}, called
\function{initialize} and \function{deinitialize}. A
\type{LibraryInitializer} merely provides a convenient RAII wrapper
for the operations (and thus for the internal library state as well).

\subsection{Gotchas}

There are a few things to watch out for to prevent problems when using Botan.

Never allocate any kind of Botan object globally. The problem with doing this
is that the constructor for such an object will be called before the library is
initialized. Many Botan objects will, in their constructor, make one or more
calls into the library global state object. Access to this object is checked,
so an exception should be thrown (rather than a memory access violation or
undetected uninitialized object access). A rough equivalent which will work is
to keep a global pointer to the object, initializing it after creating your
\type{LibraryInitializer}. Merely making the \type{LibraryInitializer} also
global will probably not help, because C++ does not make very strong guarantees
about the order that such objects will be created.

The same rule applies for making sure the destructors of all your Botan objects
are called before the \type{LibraryInitializer} is destroyed. This implies you
can't have static variables that are Botan objects inside functions or classes
(since in most C++ runtimes, these objects will be destroyed after main has
returned). This is inelegant, but seems to not cause many problems in practice.

Botan's memory object classes (\type{MemoryVector},
\type{SecureVector}, \type{SecureBuffer}) are extremely primitive, and
do not meet the requirements for an STL container object. After Botan
starts adopting C++0x features, they will be replaced by typedefs of
\type{std::vector} with a custom allocator.

Prefer using the factory methods to creating objects directly on the
stack. This helps insulate your code against changes in the
implementation, and using a late binding allows your code to access
faster implementations (hardware or faster software) that might be
detected as available at runtime.

Use a \function{try}/\function{catch} block inside your
\function{main} function, and catch any \type{std::exception} throws
(remember to catch by reference, as \type{std::exception}'s
\function{what} method is polymorphic). This is not strictly required,
but if you don't, and Botan throws an exception, the runtime will call
\function{std::terminate}, which usually calls \function{abort} or
something like it, leaving you (or worse, a user of your application)
wondering what went wrong.

\subsection{Information Flow: Pipes and Filters}

Many common uses of cryptography involve processing one or more
streams of data (be it from sockets, files, or a hardware device).
Botan provides services which make setting up data flows through
various operations, such as compression, encryption, and base64
encoding. Each of these operations is implemented in what are called
\emph{filters} in Botan. A set of filters are created and placed into
a \emph{pipe}, and information ``flows'' through the pipe until it
reaches the end, where the output is collected for retrieval. If
you're familiar with the Unix shell environment, this design will
sound quite familiar.

Here is an example which uses a pipe to base64 encode some strings:

\begin{verbatim}
  Pipe pipe(new Base64_Encoder); // pipe owns the pointer
  pipe.start_msg();
  pipe.write(``message 1''); 
  pipe.end_msg(); // flushes buffers, increments message number

  // process_msg(x) is start_msg() && write(x) && end_msg()
  pipe.process_msg(``message2'');

  std::string m1 = pipe.read_all_as_string(0); // ``message1''
  std::string m2 = pipe.read_all_as_string(1); // ``message2''
\end{verbatim}

Bytestreams in the pipe are grouped into messages; blocks of data that
are processed in an identical fashion (\ie, with the same sequence of
\type{Filter}s). Messages are delimited by calls to
\function{start\_msg} and \function{end\_msg}. Each message in a pipe
has its own number, which increments starting from zero.

As you can see, the \type{Base64\_Encoder} was allocated using
\keyword{new}; but where was it deallocated? When a filter object is
passed to a \type{Pipe}, the pipe takes ownership of the object, and
will deallocate it when it is no longer needed.

There are two different ways to make use of messages. One is to send
several messages through a \type{Pipe} without changing the \type{Pipe}'s
configuration, so you end up with a sequence of messages; one use of this would
be to send a sequence of identically encrypted UDP packets, for example (note
that the \emph{data} need not be identical; it is just that each is encrypted,
encoded, signed, etc in an identical fashion). Another is to change the filters
that are used in the \type{Pipe} between each message, by adding or removing
\type{Filter}s; functions that let you do this are documented in the Pipe API
section.

Most operations in Botan have a corresponding filter for use in Pipe.
Here's code that encrypts a string with AES-128 in CBC mode:

\begin{verbatim}
  SymmetricKey key(16); // a random 128-bit key
  InitializationVector iv(16); // a random 128-bit IV

  // Notice the algorithm we want is specified by a string
  Pipe pipe(get_cipher(``AES-128/CBC'', key, iv, ENCRYPTION));

  pipe.process_msg(``secrets'');
  pipe.process_msg(``more secrets'');

  MemoryVector<byte> c1 = pipe.read_all(0);

  byte c2[4096] = { 0 };
  u32bit got_out = pipe.read(c2, sizeof(c2), 1);
  // use c2[0...got_out]
\end{verbatim}

\type{Pipe} also has convenience methods for dealing with
\type{std::iostream}s. Here is an example of those, using
the \type{Bzip\_Compression} filter (included as a module;
if you have bzlib available, check \filename{building.pdf}
for how to enable it) to compress a file:

\begin{verbatim}
  std::ifstream in(``data.bin'', std::ios::binary)
  std::ofstream out(``data.bin.bz2'', std::ios::binary)

  Pipe pipe(new Bzip_Compression);

  pipe.start_msg();
  in >> pipe;
  pipe.end_msg();
  out << pipe;
\end{verbatim}

However there is a hitch to the code above; the complete contents of
the compressed data will be held in memory until the entire message
has been compressed, at which time the statement \verb|out << pipe| is
executed, and the data is freed as it is read from the pipe and
written to the file. But if the file is very large, we might not have
enough physical memory (or even enough virtual memory!) for that to be
practical. So instead of storing the compressed data in the pipe for
reading it out later, we divert it directly to the file:

\begin{verbatim}
  std::ifstream in(``data.bin'', std::ios::binary)
  std::ofstream out(``data.bin.bz2'', std::ios::binary)

  Pipe pipe(new Bzip_Compression, new DataSink_Stream(out));

  pipe.start_msg();
  in >> pipe;
  pipe.end_msg();
\end{verbatim}

This is the first code we've seen so far that uses more than one
filter in a pipe. The output of the compressor is sent to the
\type{DataSink\_Stream}. Anything written to a \type{DataSink\_Stream}
is written to a file; the filter produces no output. As soon as the
compression algorithm finishes up a block of data, it will send it along,
at which point it will immediately be written to disk; if you were to
call \verb|pipe.read_all()| after \verb|pipe.end_msg()|, you'd get an
empty vector out.

Here's an example using two computational filters:

\begin{verbatim}
   SymmetricKey key(32);
   InitializationVector iv(16); // or use: block_size_of("AES")
   Pipe encryptor(get_cipher("AES/CBC/PKCS7", key, iv, ENCRYPTION),
                  new Base64_Encoder);
   encryptor.start_msg();
   file >> encryptor;
   encryptor.end_msg(); // flush buffers, complete computations
   std::cout << encryptor;
\end{verbatim}

\subsection{Fork}

It is fairly common that you might receive some data and want to perform more
than one operation on it (\ie, encrypt it with DES and calculate the MD5 hash
of the plaintext at the same time). That's where \type{Fork} comes
in. \type{Fork} is a filter that takes input and passes it on to \emph{one or
more} \type{Filter}s which are attached to it. \type{Fork} changes the nature
of the pipe system completely. Instead of being a linked list, it becomes a
tree.

Each \type{Filter} in the fork is given its own output buffer, and
thus its own message. For example, if you had previously written two
messages into a \type{Pipe}, then you start a new one with a
\type{Fork} which has three paths of \type{Filter}'s inside it, you
add three new messages to the \type{Pipe}. The data you put into the
\type{Pipe} is duplicated and sent into each set of \type{Filter}s,
and the eventual output is placed into a dedicated message slot in the
\type{Pipe}.

Messages in the \type{Pipe} are allocated in a depth-first manner. This is only
interesting if you are using more than one \type{Fork} in a single \type{Pipe}.
As an example, consider the following:

\begin{verbatim}
   Pipe pipe(new Fork(
                new Fork(
                   new Base64_Encoder,
                   new Fork(
                      NULL,
                      new Base64_Encoder
                      )
                   ),
                new Hex_Encoder
                )
      );
\end{verbatim}

In this case, message 0 will be the output of the first \type{Base64\_Encoder},
message 1 will be a copy of the input (see below for how \type{Fork} interprets
NULL pointers), message 2 will be the output of the second
\type{Base64\_Encoder}, and message 3 will be the output of the
\type{Hex\_Encoder}. As you can see, this results in message numbers being
allocated in a top to bottom fashion, when looked at on the screen. However,
note that there could be potential for bugs if this is not anticipated. For
example, if your code is passed a \type{Filter}, and you assume it is a
``normal'' one which only uses one message, your message offsets would be
wrong, leading to some confusion during output.

If Fork's first argument is a null pointer, but a later argument is
not, then Fork will feed a copy of its input directly through. Here's
a case where that is useful:

\begin{verbatim}
   // have std::string ciphertext, auth_code, key, iv, mac_key;

   Pipe pipe(new Base64_Decoder, get_cipher(``AES-128'', key, iv, DECRYPTION),
             new Fork(
                0
                new MAC_Filter(``HMAC(SHA-1)'', mac_key)
             )
      );

   pipe.process_msg(ciphertext);
   std::string plaintext = pipe.read_all_as_string(0);
   SecureVector<byte> mac = pipe.read_all(1);

   if(mac != auth_code)
      error();
\end{verbatim}

Here we wanted to not only decrypt the message, but send the decrypted
text through an additional computation, in order to compute the
authentication code.

Any \type{Filter}s which are attached to the \type{Pipe} after the
\type{Fork} are implicitly attached onto the first branch created by
the fork. For example, let's say you created this \type{Pipe}:

\begin{verbatim}
Pipe pipe(new Fork(new Hash_Filter("MD5"), new Hash_Filter("SHA-1")),
          new Hex_Encoder);
\end{verbatim}

And then called \function{start\_msg}, inserted some data, then
\function{end\_msg}. Then \arg{pipe} would contain two messages. The
first one (message number 0) would contain the MD5 sum of the input in
hex encoded form, and the other would contain the SHA-1 sum of the
input in raw binary. However, it's much better to use a \type{Chain}
instead.

\subsubsection{Chain}

A \type{Chain} filter creates a chain of \type{Filter}s and
encapsulates them inside a single filter (itself). This allows a
sequence of filters to become a single filter, to be passed into or
out of a function, or to a \type{Fork} constructor.

You can call \type{Chain}'s constructor with up to 4 \type{Filter*}s
(they will be added in order), or with an array of \type{Filter*}s and
a \type{u32bit} which tells \type{Chain} how many \type{Filter*}s are
in the array (again, they will be attached in order). Here's the
example from the last section, using chain instead of relying on the
obscure rule that version used.

\begin{verbatim}
  Pipe pipe(new Fork(
                new Chain(new Hash_Filter("MD5"), new Hex_Encoder),
                new Hash_Filter("SHA-1")
                )
           );
\end{verbatim}

\subsection{The Pipe API}

\subsubsection{Initializing Pipe}

By default, \type{Pipe} will do nothing at all; any input placed into
the \type{Pipe} will be read back unchanged. Obviously, this has
limited utility, and presumably you want to use one or more
\type{Filter}s to somehow process the data. First, you can choose a
set of \type{Filter}s to initialize the \type{Pipe} with via the
constructor. You can pass it either a set of up to 4 \type{Filter*}s,
or a pre-defined array and a length:

\begin{verbatim}
   Pipe pipe1(new Filter1(/*args*/), new Filter2(/*args*/),
              new Filter3(/*args*/), new Filter4(/*args*/));
   Pipe pipe2(new Filter1(/*args*/), new Filter2(/*args*/));

   Filter* filters[5] = {
     new Filter1(/*args*/), new Filter2(/*args*/), new Filter3(/*args*/),
     new Filter4(/*args*/), new Filter5(/*args*/) /* more if desired... */
   };
   Pipe pipe3(filters, 5);
\end{verbatim}

This is by far the most common way to initialize a \type{Pipe}. However,
occasionally a more flexible initialization strategy is necessary; this is
supported by 4 member functions: \function{prepend}(\type{Filter*}),
\function{append}(\type{Filter*}), \function{pop}(), and \function{reset}().
These functions may only be used while the \type{Pipe} in question is not in
use; that is, either before calling \function{start\_msg}, or after
\function{end\_msg} has been called (and no new calls to \function{start\_msg}
have been made yet).

The function \function{reset}() simply removes all the \type{Filter}s
which the \type{Pipe} is currently using~--~it is reset to an
initialize, ``empty'' state.  Any data which is being retained by the
\type{Pipe} is retained after a \function{reset}(), and
\function{reset}() does not affect the message numbers (discussed
later).

Calling \function{prepend} and \function{append} will either prepend
or append the passed \type{Filter} object to the list of
transformations. For example, if you \function{prepend} a
\type{Filter} implementing encryption, and the \type{Pipe} already had
a \type{Filter} which hex encoded the input, then the next set of
input would be first encrypted, then hex encoded. Alternately, if you
called \function{append}, then the input would be first be hex
encoded, and then encrypted (which is not terribly useful in this
particular example).

Finally, calling \function{pop}() will remove the first transformation
of the \type{Pipe}. Say we had called \function{prepend} to put an
encryption \type{Filter} into a \type{Pipe}; calling \function{pop}()
would remove this \type{Filter} and return the \type{Pipe} to its
state before we called \function{prepend}.

\subsubsection{Giving Data to a Pipe}

Input to a \type{Pipe} is delimited into messages, which can be read from
independently (\ie, you can read 5 bytes from one message, and then all of
another message, without either read affecting any other messages). The
messages are delimited by calls to \function{start\_msg} and
\function{end\_msg}. In between these two calls, you can write data into a
\type{Pipe}, and it will be processed by the \type{Filter}(s) that it
contains. Writes at any other time are invalid, and will result in an
exception.

As to writing, you can call any of the functions called \function{write}(),
which can take any of: a \type{byte[]}/\type{u32bit} pair, a
\type{SecureVector<byte>}, a \type{std::string}, a \type{DataSource\&}, or a
single \type{byte}.

Sometimes, you may want to do only a single write per message. In this case,
you can use the \function{process\_msg} series of functions, which start a
message, write their argument into the \type{Pipe}, and then end the
message. In this case you would not make any explicit calls to
\function{start\_msg}/\function{end\_msg}. The version of \function{write}
which takes a single \type{byte} is not supported by \function{process\_msg},
but all the other variants are.

\type{Pipe} can also be used with the \verb|>>| operator, and will accept a
\type{std::istream}, (or on Unix systems with the \verb|fd_unix| module), a
Unix file descriptor. In either case, the entire contents of the file will be
read into the \type{Pipe}.

\subsubsection{Getting Output from a Pipe}

Retrieving the processed data from a \type{Pipe} is a bit more complicated, for
various reasons. In particular, because \type{Pipe} will separate each message
into a separate buffer, you have to be able to retrieve data from each message
independently. Each of \type{Pipe}'s read functions has a final parameter which
specifies what message to read from (as a 32-bit integer). If this parameter is
set to \type{Pipe::DEFAULT\_MESSAGE}, it will read the current default message
(\type{DEFAULT\_MESSAGE} is also the default value of this parameter). The
parameter will not be mentioned in further discussion of the reading API, but
it is always there (unless otherwise noted).

Reading is done with a variety of functions. The most basic are \type{u32bit}
\function{read}(\type{byte} \arg{out}[], \type{u32bit} \arg{len}) and
\type{u32bit} \function{read}(\type{byte\&} \arg{out}). Each reads into
\arg{out} (either up to \arg{len} bytes, or a single byte for the one taking a
\type{byte\&}), and returns the total number of bytes read. There is a variant
of these functions, all named \function{peek}, which performs the same
operations, but does not remove the bytes from the message (reading is a
destructive operation with a \type{Pipe}).

There are also the functions \type{SecureVector<byte>} \function{read\_all}(),
and \type{std::string} \function{read\_all\_as\_string}(), which return the
entire contents of the message, either as a memory buffer, or a
\type{std::string} (which is generally only useful is the \type{Pipe} has
encoded the message into a text string, such as when a \type{Base64\_Encoder}
is used).

To determine how many bytes are left in a message, call \type{u32bit}
\function{remaining}() (which can also take an optional message
number). Finally, there are some functions for managing the default message
number: \type{u32bit} \function{default\_msg}() will return the current default
message, \type{u32bit} \function{message\_count}() will return the total number
of messages (0...\function{message\_count}()-1), and
\function{set\_default\_msg}(\type{u32bit} \arg{msgno}) will set a new default
message number (which must be a valid message number for that \type{Pipe}). The
ability to set the default message number is particularly important in the case
of using the file output operations (\verb|<<| with a \type{std::ostream} or
Unix file descriptor), because there is no way to specify it explicitly when
using the output operator.

\subsection{A Filter Example}

Here is some code which takes one or more filenames in \arg{argv} and
calculates the result of several hash functions for each file. The complete
program can be found as \filename{hasher.cpp} in the Botan distribution. For
brevity, most error checking has been removed.

\begin{verbatim}
   string name[3] = { "MD5", "SHA-1", "RIPEMD-160" };
   Botan::Filter* hash[3] = {
      new Botan::Chain(new Botan::Hash_Filter(name[0]),
                        new Botan::Hex_Encoder),
      new Botan::Chain(new Botan::Hash_Filter(name[1]),
                        new Botan::Hex_Encoder),
      new Botan::Chain(new Botan::Hash_Filter(name[2]),
                        new Botan::Hex_Encoder) };

   Botan::Pipe pipe(new Botan::Fork(hash, COUNT));

   for(u32bit j = 1; argv[j] != 0; j++)
      {
      ifstream file(argv[j]);
      pipe.start_msg();
      file >> pipe;
      pipe.end_msg();
      file.close();
      for(u32bit k = 0; k != 3; k++)
         {
         pipe.set_default_msg(3*(j-1)+k);
         cout << name[k] << "(" << argv[j] << ") = " << pipe << endl;
         }
      }
\end{verbatim}


\subsection{Filter Catalog}

This section contains descriptions of every \type{Filter} included in
the portable sections of Botan. \type{Filter}s provided by modules
are documented elsewhere.

\subsubsection{Keyed Filters}

A few sections ago, it was mentioned that \type{Pipe} can process multiple
messages, treating each of them exactly the same. Well, that was a bit of a
lie. There are some algorithms (in particular, block ciphers not in ECB mode,
and all stream ciphers) that change their state as data is put through them.

Naturally, you might well want to reset the keys or (in the case of block
cipher modes) IVs used by such filters, so multiple messages can be processed
using completely different keys, or new IVs, or new keys and IVs, or whatever.
And in fact, even for a MAC or an ECB block cipher, you might well want to
change the key used from message to message.

Enter \type{Keyed\_Filter}, which acts as an abstract interface for
any filter that is uses keys: block cipher modes, stream ciphers,
MACs, and so on. It has two functions, \function{set\_key} and
\function{set\_iv}. Calling \function{set\_key} will, naturally, set
(or reset) the key used by the algorithm. Setting the IV only makes
sense in certain algorithms -- a call to \function{set\_iv} on an
object that doesn't support IVs will be ignored. You \emph{must} call
\function{set\_key} before calling \function{set\_iv}: while not all
\type{Keyed\_Filter} objects require this, you should assume it is
required anytime you are using a \type{Keyed\_Filter}.

Here's a example:

\begin{verbatim}
   Keyed_Filter *cast, *hmac;
   Pipe pipe(new Base64_Decoder,
             // Note the assignments to the cast and hmac variables
             cast = new CBC_Decryption("CAST-128", "PKCS7", cast_key, iv),
             new Fork(
                0, // Read the section 'Fork' to understand this
                new Chain(
                   hmac = new MAC_Filter("HMAC(SHA-1)", mac_key, 12),
                   new Base64_Encoder
                   )
                )
      );
   pipe.start_msg();
   [use pipe for a while, decrypt some stuff, derive new keys and IVs]
   pipe.end_msg();

   cast->set_key(cast_key2);
   cast->set_iv(iv2);
   hmac->set_key(mac_key2);

   pipe.start_msg();
   [use pipe for some other things]
   pipe.end_msg();
\end{verbatim}

There are some requirements to using \type{Keyed\_Filter} which you must
follow. If you call \function{set\_key} or \function{set\_iv} on a filter which
is owned by a \type{Pipe}, you must do so while the \type{Pipe} is
``unlocked''. This refers to the times when no messages are being processed by
\type{Pipe} -- either before \type{Pipe}'s \function{start\_msg} is called, or
after \function{end\_msg} is called (and no new call to \function{start\_msg}
has happened yet). Doing otherwise will result in undefined behavior, probably
silently getting invalid output.

And remember: if you're resetting both values, reset the key \emph{first}.

\subsubsection{Cipher Filters}

Getting ahold of a \type{Filter} implementing a cipher is very easy. Simply
make sure you're including the header \filename{lookup.h}, and call
\function{get\_cipher}. Generally you will pass the return value directly into
a \type{Pipe}. There are actually a couple different functions, which do pretty
much the same thing:

\function{get\_cipher}(\type{std::string} \arg{cipher\_spec},
                       \type{SymmetricKey} \arg{key},
                       \type{InitializationVector} \arg{iv},
                       \type{Cipher\_Dir} \arg{dir});

\function{get\_cipher}(\type{std::string} \arg{cipher\_spec},
                       \type{SymmetricKey} \arg{key},
                       \type{Cipher\_Dir} \arg{dir});

The version that doesn't take an IV is useful for things that don't use them,
like block ciphers in ECB mode, or most stream ciphers. If you specify a
\arg{cipher\_spec} that does want a IV, and you use the version that doesn't
take one, an exception will be thrown. The \arg{dir} argument can be either
\type{ENCRYPTION} or \type{DECRYPTION}. In a few cases, like most (but not all)
stream ciphers, these are equivalent, but even then it provides a way of
showing the ``intent'' of the operation to readers of your code.

The \arg{cipher\_spec} is a string that specifies what cipher is to be
used. The general syntax for \arg{cipher\_spec} is ``STREAM\_CIPHER'',
``BLOCK\_CIPHER/MODE'', or ``BLOCK\_CIPHER/MODE/PADDING''. In the case of
stream ciphers, no mode is necessary, so just the name is sufficient. A block
cipher requires a mode of some sort, which can be ``ECB'', ``CBC'', ``CFB(n)'',
``OFB'', ``CTR-BE'', or ``EAX(n)''. The argument to CFB mode is how many bits
of feedback should be used. If you just use ``CFB'' with no argument, it will
default to using a feedback equal to the block size of the cipher. EAX mode
also takes an optional bit argument, which tells EAX how large a tag size to
use~--~generally this is the size of the block size of the cipher, which is the
default if you don't specify any argument.

In the case of the ECB and CBC modes, a padding method can also be
specified. If it is not supplied, ECB defaults to not padding, and CBC defaults
to using PKCS \#5/\#7 compatible padding. The padding methods currently
available are ``NoPadding'', ``PKCS7'', ``OneAndZeros'', and ``CTS''. CTS
padding is currently only available for CBC mode, but the others can also be
used in ECB mode.

Some example \arg{cipher\_spec} arguments are: ``DES/CFB(32)'',
``TripleDES/OFB'', ``Blowfish/CBC/CTS'', ``SAFER-SK(10)/CBC/OneAndZeros'',
``AES/EAX'', ``ARC4''

``CTR-BE'' refers to counter mode where the counter is incremented as if it
were a big-endian encoded integer. This is compatible with most other
implementations, but it is possible some will use the incompatible little
endian convention. This version would be denoted as ``CTR-LE'' if it were
supported.

``EAX'' is a new cipher mode designed by Wagner, Rogaway, and Bellare. It is an
authenticated cipher mode (that is, no separate authentication is needed), has
provable security, and is free from patent entanglements. It runs about half as
fast as most of the other cipher modes (like CBC, OFB, or CTR), which is not
bad considering you don't need to use an authentication code.

\subsubsection{Hashes and MACs}

Hash functions and MACs don't need anything special when it comes to
filters. Both just take their input and produce no output until
\function{end\_msg()} is called, at which time they complete the hash or MAC
and send that as output.

These \type{Filter}s take a string naming the type to be used. If for some
reason you name something that doesn't exist, an exception will be thrown.

\noindent
\function{Hash\_Filter}(\type{std::string} \arg{hash},
                        \type{u32bit} \arg{outlength}):

This type hashes its input with \arg{hash}. When \function{end\_msg} is called
on the owning \type{Pipe}, the hash is completed and the digest is sent on to
the next thing in the pipe. The argument \arg{outlength} specifies how much of
the output of the hash will be passed along to the next filter when
\function{end\_msg} is called. By default, it will pass the entire hash.

Examples of names for \function{Hash\_Filter} are ``SHA-1'' and ``Whirlpool''.

\noindent
\function{MAC\_Filter}(\type{std::string} \arg{mac},
                       \type{const SymmetricKey\&} \arg{key},
                       \type{u32bit} \arg{outlength}):

The constructor for a \type{MAC\_Filter} takes a key, used in calculating the
MAC, and a length parameter, which has semantics exactly the same as the one
passed to \type{Hash\_Filter}s constructor.

Examples for \arg{mac} are ``HMAC(SHA-1)'', ``CMAC(AES-128)'', and the
exceptionally long, strange, and probably useless name
``CMAC(Lion(Tiger(20,3),MARK-4,1024))''.

\subsubsection{PK Filters}

There are four classes in this category, \type{PK\_Encryptor\_Filter},
\type{PK\_Decryptor\_Filter}, \type{PK\_Signer\_Filter}, and
\type{PK\_Verifier\_Filter}. Each takes a pointer to an object of the
appropriate type (\type{PK\_Encryptor}, \type{PK\_Decryptor}, etc) which is
deleted by the destructor. These classes are found in \filename{pk\_filts.h}.

Three of these, for encryption, decryption, and signing are pretty much
identical conceptually. Each of them buffers its input until the end of the
message is marked with a call to the \function{end\_msg} function. Then they
encrypt, decrypt, or sign their input and send the output (the ciphertext, the
plaintext, or the signature) into the next filter.

Signature verification works a little differently, because it needs to know
what the signature is in order to check it. You can either pass this in along
with the constructor, or call the function \function{set\_signature} -- with
this second method, you need to keep a pointer to the filter around so you can
send it this command. In either case, after \function{end\_msg} is called, it
will try to verify the signature (if the signature has not been set by either
method, an exception will be thrown here). It will then send a single byte onto
the next filter -- a 1 or a 0, which specifies whether the signature verified
or not (respectively).

For more information about PK algorithms (including creating the appropriate
objects to pass to the constructors), read the section ``Public Key
Cryptography'' in this manual.

\subsubsection{Encoders}

Often you want your data to be in some form of text (for sending over channels
which aren't 8-bit clean, printing it, etc). The filters \type{Hex\_Encoder}
and \type{Base64\_Encoder} will convert arbitrary binary data into hex or
base64 formats. Not surprisingly, you can use \type{Hex\_Decoder} and
\type{Base64\_Decoder} to convert it back into its original form.

Both of the encoders can take a few options about how the data should be
formatted (all of which have defaults). The first is a \type{bool} which simply
says if the encoder should insert line breaks. This defaults to
false. Line breaks don't matter either way to the decoder, but it makes the
output a bit more appealing to the human eye, and a few transport mechanisms
(notably some email systems) limit the maximum line length.

The second encoder option is an integer specifying how long such lines will be
(obviously this will be ignored if line-breaking isn't being used). The default
tends to be in the range of 60-80 characters, but is not specified exactly. If
you want a specific value, set it. Otherwise the default should be fine.

Lastly, \type{Hex\_Encoder} takes an argument of type \type{Case}, which can be
\type{Uppercase} or \type{Lowercase} (default is \type{Uppercase}). This
specifies what case the characters A-F should be output as. The base64 encoder
has no such option, because it uses both upper and lower case letters for its
output.

The decoders both take a single option, which tells it how the object should
behave in the case of invalid input. The enum (called \type{Decoder\_Checking})
can take on any of three values: \type{NONE}, \type{IGNORE\_WS}, and
\type{FULL\_CHECK}. With \type{NONE} (the default, for compatibility with
previous releases), invalid input (for example, a ``z'' character in supposedly
hex input) will simply be ignored. With \type{IGNORE\_WS}, whitespace will be
ignored by the decoder, but receiving other non-valid data will raise an
exception. Finally, \type{FULL\_CHECK} will raise an exception for \emph{any}
characters not in the encoded character set, including whitespace.

You can find the declarations for these types in \filename{hex.h} and
\filename{base64.h}.

\subsection{Rolling Your Own}

The system of filters and pipes was designed in an attempt to make it
as simple as possible to write new \type{Filter} objects. There are
essentially four functions that need to be implemented by an object
deriving from \type{Filter}:

\noindent
\type{void} \function{write}(\type{byte} \arg{input}[], \type{u32bit}
\arg{length}):

The \function{write} function is what is called when a filter receives input
for it to process. The filter is \emph{not} required to process it right away;
many filters buffer their input before producing any output. A filter will
usually have \function{write} called many times during its lifetime.

\noindent
\type{void} \function{send}(\type{byte} \arg{output}[], \type{u32bit}
\arg{length}):

Eventually, a filter will want to produce some output to send along to the next
filter in the pipeline. It does so by calling \function{send} with whatever it
wants to send along to the next filter. There is also a version of
\function{send} taking a single byte argument, as a convenience.

\noindent
\type{void} \function{start\_msg()}:

This function is optional. Implement it if your \type{Filter} would like to do
some processing or setup at the start of each message (for an example, see the
Zlib compression module).

\noindent
\type{void} \function{end\_msg()}:

Implementing the \function{end\_msg} function is optional. It is called when it
has been requested that filters finish up their computations. Note that they
must \emph{not} deallocate their resources; this should be done by their
destructor. They should simply finish up with whatever computation they have
been working on (for example, a compressing filter would flush the compressor
and \function{send} the final block), and empty any buffers in preparation for
processing a fresh new set of input. It is essentially the inverse of
\function{start\_msg}.

Additionally, if necessary, filters can define a constructor that takes any
needed arguments, and a destructor to deal with deallocating memory, closing
files, etc.

There is also a \type{BufferingFilter} class (in \filename{buf\_filt.h}) which
will take a message and split it up into an initial block which can be of any
size (including zero), a sequence of fixed sized blocks of any non-zero size,
and last (possibly zero-sized) final block. This might make a useful base class
for your filters, depending on what you have in mind.


\pagebreak
\section{Public Key Cryptography}

Let's create an RSA private key:

\begin{verbatim}
   RSA_PrivateKey priv_rsa(1024 /* bits */);
\end{verbatim}

We can easily turn this into a public key, which we can then send to
someone:
 
\begin{verbatim}
   RSA_PublicKey pub_rsa = priv_rsa;
\end{verbatim}




\subsection{Creating PK Algorithm Key Objects}

The library has interfaces for encryption, signatures, etc that do not require
knowing the exact algorithm in use (for example RSA and Rabin-Williams
signatures are handled by the exact same code path).

One place where we \emph{do} need to know exactly what kind of algorithm is in
use is when we are creating a key (\emph{But}: read the section ``Importing and
Exporting PK Keys'', later in this manual).

There are (currently) two kinds of public key algorithms in Botan: ones based
on integer factorization (RSA and Rabin-Williams), and ones based on the
discrete logarithm problem (DSA, Diffie-Hellman, Nyberg-Rueppel, and
ElGamal). Since discrete logarithm parameters (primes and generators) can be
shared among many keys, there is the notion of these being a combined type
(called \type{DL\_Group}).

There are two ways to create a DL private key (such as
\type{DSA\_PrivateKey}). One is to pass in just a \type{DL\_Group} object -- a
new key will automatically be generated. The other involves passing in a group
to use, along with both the public and private values (private value first).

Since in integer factorization algorithms, the modulus used isn't shared by
other keys, we don't use this notion. You can create a new key by passing in a
\type{u32bit} telling how long (in bits) the key should be, or you can copy an
pre-existing key by passing in the appropriate parameters (primes, exponents,
etc). For RSA and Rabin-Williams (the two IF schemes in Botan), the parameters
are all \type{BigInt}s: prime 1, prime 2, encryption exponent, decryption
exponent, modulus. The last two are optional, since they can easily be derived
from the first three.

\subsubsection{Creating a DL\_Group}

There are quite a few ways to get a \type{DL\_Group} object. The best is to use
the function \function{get\_dl\_group}, which takes a string naming a group; it
will either return that group, if it knows about it, or throw an
exception. Names it knows about include ``IETF-n'' where n is 768, 1024, 1536,
2048, 3072, or 4096, and ``DSA-n'', where n is 512, 768, or 1024. The IETF
groups are the ones specified for use with IPSec, and the DSA ones are the
default DSA parameters specified by Java's JCE. For DSA and Nyberg-Rueppel, you
should only use the ``DSA-n'' groups, while Diffie-Hellman and ElGamal can use
either type (keep in mind that some applications/standards require DH/ELG to
use DSA-style primes, while others require strong prime groups).

You can also generate a new random group. This is not recommend, because it is
quite slow, especially for safe primes.

You can register a new DL group with \function{add\_dl\_group} with a string
naming the group and the \type{DL\_Group}. Future lookups on that name will
return the group. There is no reason to register the group if you do decide to
use a distinct DL group for each key.

\subsection{Key Checking}

Most public key algorithms have limitations or restrictions on their
parameters. For example RSA requires an odd exponent, and algorithms based on
the discrete logarithm problem need a generator $> 1$.

Each low-level public key type has a function named \function{check\_key} which
takes a \type{bool}. This function returns a boolean value that declares
whether or not the key is valid (from an algorithmic standpoint). For example,
it will check to make sure that the prime parameters of a DSA key are, in fact,
prime. It does not have anything to do with the validity of the key for any
particular use, nor does it have anything to do with certificates which link a
key (which, after all, is just some numbers) with a user or other entity. If
\function{check\_key}'s argument is \type{true}, then it does ``strong''
checking, which includes fairly expensive operations like primality checking.

Keys are always checked when they are loaded or generated, so typically there
is no reason to use this function directly. However, you can disable or reduce
the checks for particular cases (public keys, loaded private keys, generated
private keys) by setting the right config toggle (see the section on the
configuration subsystem for details).

\subsection{Getting a PK algorithm object}

The key types, like \type{RSA\_PrivateKey}, do not implement any kind of
padding or encoding (which is generally necessary for security). To get an
object like this, the easiest thing to do is call the functions found in
\filename{look\_pk.h}. Generally these take a key, followed by a string that
specified what hashing and encoding method(s) to use. Examples of such strings
are ``EME1(SHA-1)'' for OAEP encryption and ``EMSA4(SHA-1)'' for PSS signatures
(where the message is hashed using SHA-1).

Here are some basic examples (using an RSA key) to give you a feel for the
possibilities. These examples assume \type{rsakey} is an
\type{RSA\_PrivateKey}, since otherwise we would not be able to create a
decryption or signature object with it (you can create encryption or signature
verification objects with public keys, naturally). Remember to delete these
objects when you're done with them.

\begin{verbatim}
   // PKCS #1 v2.0 / IEEE 1363 compatible encryption
   PK_Encryptor* rsa_enc1 = get_pk_encryptor(rsakey, "EME1(RIPEMD-160)");
   // PKCS #1 v1.5 compatible encryption
   PK_Encryptor* rsa_enc2 = get_pk_encryptor(rsakey, "PKCS1v15");

   // Raw encryption: no padding, input is directly encrypted by the key
   // Don't use this unless you know what you're doing
   PK_Encryptor* rsa_enc3 = get_pk_encryptor(rsakey, "Raw");

   // This object can decrypt things encrypted by rsa_enc1
   PK_Decryptor* rsa_dec1 = get_pk_decryptor(rsakey, "EME1(RIPEMD-160)");

   // PKCS #1 v1.5 compatible signatures
   PK_Signer* rsa_sig = get_pk_signer(rsakey, "EMSA3(MD5)");
   PK_Verifier* rsa_verify = get_pk_verifier(rsakey, "EMSA3(MD5)");

   // PKCS #1 v2.1 compatible signatures
   PK_Signer* rsa_sig2 = get_pk_signer(rsakey, "EMSA4(SHA-1)");
   PK_Verifier* rsa_verify2 = get_pk_verifier(rsakey, "EMSA4(SHA-1)");

   // Hash input with SHA-1, but don't pad the input in any way; usually
   // used with DSA/NR, not RSA
   PK_Signer* rsa_sig = get_pk_signer(rsakey, "EMSA1(SHA-1)");
\end{verbatim}

\subsection{Encryption}

The \type{PK\_Encryptor} and \type{PK\_Decryptor} classes are the interface for
encryption and decryption, respectively.

Calling \function{encrypt} with a \type{byte} array and a length parameter will
return the input encrypted with whatever scheme is being used. Calling the
similar \function{decrypt} will perform the inverse operation. You can also do
these operations with \type{SecureVector<byte>}s. In all cases, the output is
returned via a \type{SecureVector<byte>}.

If you attempt an operation with a larger size than the key can support (this
limit varies based on the algorithm, the key size, and the padding method used
(if any)), an exception will be thrown. Alternately, you can call
\function{maximum\_input\_size}, which will return the maximum size you can
safely encrypt. In fact, you can often encrypt an object that is one byte
longer, but only if enough of the high bits of the leading byte are set to
zero. Since this is pretty dicey, it's best to stick with the advertised
maximum.

Available public key encryption algorithms in Botan are RSA and ElGamal. The
encoding methods are EME1, denoted by ``EME1(HASHNAME)'', PKCS \#1 v1.5,
called ``PKCS1v15'' or ``EME-PKCS1-v1\_5'', and raw encoding (``Raw'').

For compatibility reasons, PKCS \#1 v1.5 is recommend for use with ElGamal
(most other implementations of ElGamal do not support any other encoding
format). RSA can also be used with PKCS \# 1 encoding, but because of various
possible attacks, EME1 is the preferred encoding. EME1 requires the use of a
hash function: unless a competent applied cryptographer tells you otherwise,
you should use SHA-1.

Don't use ``Raw'' encoding unless you need it for backward compatibility with
old protocols. There are many possible attacks against both ElGamal and RSA
when they are used this way.

\subsection{Signatures}

The signature algorithms look quite a bit like the hash functions. You can
repeatedly call \function{update}, giving more and more of a message you wish
to sign, and then call \function{signature}, which will return a signature for
that message. If you want to do it all in one shot, call
\function{sign\_message}, which will just call \function{update} with its
argument and then return whatever \function{signature} returns.

You can validate a signature by updating the verifier class, and finally seeing
the if the value returned from \function{check\_signature} is true (you pass
the supposed signature to the \function{check\_signature} function as a byte
array and a length or as a \type{MemoryRegion<byte>}). There is another
function, \function{verify\_message}, which takes a pair of byte array/length
pairs (or a pair of \type{MemoryRegion<byte>} objects), the first of which is
the message, the second being the (supposed) signature. It returns true if the
signature is valid and false otherwise.

Available public key signature algorithms in Botan are RSA, DSA,
Nyberg-Rueppel, and Rabin-Williams. Signature encoding methods include EMSA1,
EMSA2, EMSA3, EMSA4, and Raw. All of them, except Raw, take a parameter naming
a message digest function to hash the message with. Raw actually signs the
input directly; if the message is too big, the signing operation will fail. Raw
is not useful except in very specialized applications.

There are various interactions which make certain encoding schemes and signing
algorithms more or less useful.

EMSA2 is the usual method for encoding Rabin-William signatures, so for
compatibility with other implementations you may have to use that. EMSA4 (also
called PSS), also works with Rabin-Williams. EMSA1 and EMSA3 do \emph{not} work
with Rabin-Williams.

RSA can be used with any of the available encoding methods. EMSA4 is by far the
most secure, but is not (as of now) widely implemented. EMSA3 (also called
``EMSA-PKCS1-v1\_5'') is commonly used with RSA (for example in SSL). EMSA1
signs the message digest directly, without any extra padding or encoding. This
may be useful, but is not as secure as either EMSA3 or EMSA4. EMSA2 may be used
but is not recommended.

For DSA and Nyberg-Rueppel, you should use EMSA1. None of the other encoding
methods are particularly useful for these algorithms.

\subsection{Key Agreement}

You can get ahold of a \type{PK\_Key\_Agreement\_Scheme} object by calling
\function{get\_pk\_kas} with a key that is of a type that supports key
agreement (such as a Diffie-Hellman key stored in a \type{DH\_PrivateKey}
object), and the name of a key derivation function. This can be ``Raw'',
meaning the output of the primitive itself is returned as the key, or
``KDF1(hash)'' or ``KDF2(hash)'' where ``hash'' is any string you happen to
like (hopefully you like strings like ``SHA-1'' or ``RIPEMD-160''), or
``X9.42-PRF(keywrap)'', which uses the PRF specified in ANSI X9.42. It takes
the name or OID of the key wrap algorithm which will be used to encrypt a
content encryption key.

How key agreement generally works is that you trade public values with some
other party, and then each of you runs a computation with the other's value and
your key (this should return the same result to both parties). This computation
can be called by using \function{derive\_key} with either a byte array/length
pair, or a \type{SecureVector<byte>} than holds the public value of the other
party. The last argument to either call is a number that specifies how long a
key you want.

Depending on the key derivation function you're using, you many not
\emph{actually} get back a key of that size. In particular, ``Raw'' will return
a number about the size of the Diffie-Hellman modulus, and KDF1 can only return
a key which is the same size as the output of the hash. KDF2, on the other
hand, will always give you a key exactly as long as you request, regardless of
the underlying hash used with it. The key returned is a \type{SymmetricKey},
ready to pass to a block cipher, MAC, or other symmetric algorithm.

The public value which should be used can be obtained by calling
\function{public\_data}, which exists for any key that is associated with a
key agreement algorithm. It returns a \type{SecureVector<byte>}.

``KDF2(SHA-1)'' is by far the preferred algorithm for key derivation in new
applications. The X9.42 algorithm may be useful in some circumstances, but
unless you need X9.42 compatibility, KDF2 is easier to use.

There is a Diffie-Hellman example included in the distribution, which you may
want to examine.

\subsection{Importing and Exporting PK Keys}

[This section mentions \type{Pipe} and \type{DataSource}, which is not covered
until later in the manual. Please read those sections for more about
\type{Pipe} and \type{DataSource} and their uses.]

There are many, many different (often conflicting) standards surrounding public
key cryptography. There is, thankfully, only two major standards surrounding
the representation of a public or private key: X.509 (for public keys), and
PKCS \#8 (for private keys). Other crypto libraries, like OpenSSL and B-SAFE,
also support these formats, so you can easily exchange keys with software that
doesn't use Botan.

In addition to ``plain'' public keys, Botan also supports X.509 certificates.
These are documented in the section ``Certificate Handling'', later in this
manual.

\subsubsection{Public Keys}

The interfaces for doing either of these is quite similar. Let's look at the
X.509 stuff first:
\begin{verbatim}
namespace X509 {
   void encode(const X509_PublicKey& key, Pipe& out, X509_Encoding enc = PEM);
   std::string PEM_encode(const X509_PublicKey& out);

   X509_PublicKey* load_key(DataSource& in);
   X509_PublicKey* load_key(const std::string& file);
   X509_PublicKey* load_key(const SecureVector<byte>& buffer);
}
\end{verbatim}

Basically, \function{X509::encode} will take an \type{X509\_PublicKey}
(as of now, that's any RSA, DSA, or Diffie-Hellman key) and encodes it
using \arg{enc}, which can be either \type{PEM} or
\type{RAW\_BER}. Using \type{PEM} is \emph{highly} recommended for
many reasons, including compatibility with other software, for
transmission over 8-bit unclean channels, because it can be identified
by a human without special tools, and because it sometimes allows more
sane behavior of tools that process the data. It will place the
encoding into \arg{out}. Remember that if you have just created the
\type{Pipe} that you are passing to \function{X509::encode}, you need
to call \function{start\_msg} first. Particularly with public keys,
about 99\% of the time you just want to PEM encode the key and then
write it to a file or something. In this case, it's probably easier to
use \function{X509::PEM\_encode}. This function will simply return the
PEM encoding of the key as a \type{std::string}.

For loading a public key, the preferred method is one of the variants
of \function{load\_key}. This function will return a newly allocated
key based on the data from whatever source it is using (assuming, of
course, the source is in fact storing a representation of a public
key). The encoding used (PEM or BER) need not be specified; the format
will be detected automatically. The key is allocated with
\function{new}, and should be released with \function{delete} when you
are done with it. The first takes a generic \type{DataSource} which
you have to allocate~--~the others are simple wrapper functions that
take either a filename or a memory buffer.

So what can you do with the return value of \function{load\_key}? On
its own, a \type{X509\_PublicKey} isn't particularly useful; you can't
encrypt messages or verify signatures, or much else. But, using
\function{dynamic\_cast}, you can figure out what kind of operations
the key supports. Then, you can cast the key to the appropriate type
and pass it to a higher-level class. For example:

\begin{verbatim}
   /* Might be RSA, might be ElGamal, might be ... */
   X509_PublicKey* key = X509::load_key("pubkey.asc");
      /* You MUST use dynamic_cast to convert, because of virtual bases */
   PK_Encrypting_Key* enc_key = dynamic_cast<PK_Encrypting_Key*>(key);
   if(!enc_key)
      throw Some_Exception();
   PK_Encryptor* enc = get_pk_encryptor(*enc_key, "EME1(SHA-1)");
   SecureVector<byte> cipher = enc->encrypt(some_message, size_of_message);
\end{verbatim}

\subsubsection{Private Keys}

There are two different options for private key import/export. The first is a
plaintext version of the private key. This is supported by the following
functions:

\begin{verbatim}
namespace PKCS8 {
   void encode(const PKCS8_PrivateKey& key, Pipe& to, X509_Encoding enc = PEM);

   std::string PEM_encode(const PKCS8_PrivateKey& key);
}
\end{verbatim}

These functions are basically the same as the X.509 functions described
previously. The only difference is that they take a \type{PKCS8\_PrivateKey}
type (which, again, can be either RSA, DSA, or Diffie-Hellman, but this time
the key must be a private key). In most situations, using these is a bad idea,
because anyone can come along and grab the private key without having to know
any passwords or other secrets. Unless you have very particular security
requirements, always use the versions that encrypt the key based on a
passphrase. For importing, the same functions can be used for encrypted and
unencrypted keys.

The other way to export a PKCS \#8 key is to first encode it in the same manner
as done above, then encrypt it (using a passphrase and the techniques of PKCS
\#5), and store the whole thing into another structure. This method is
definitely preferred, since otherwise the private key is unprotected. The
following functions support this technique:

\begin{verbatim}
namespace PKCS8 {
   void encrypt_key(const PKCS8_PrivateKey& key, Pipe& out,
                    std::string passphrase, std::string pbe = "",
                    X509_Encoding enc = PEM);

   std::string PEM_encode(const PKCS8_PrivateKey& key, std::string passphrase,
                          std::string pbe = "");
}
\end{verbatim}

To export an encrypted private key, call \function{PKCS8::encrypt\_key}. The
\arg{key}, \arg{out}, and \arg{enc} arguments are similar in usage to the ones
for \function{PKCS8::encode}. As you might notice, there are two new arguments
for \function{PKCS8::encrypt\_key}, however. The first is a passphrase (which
you presumably got from a user somehow). This will be used to encrypt the key.
The second new argument is \arg{pbe}; this specifies a particular password
based encryption (or PBE) algorithm.

The \function{PEM\_encode} version shown here is similar to the one that
doesn't take a passphrase. Essentially it encrypts the key (using the default
PBE algorithm), and then returns a C++ string with the PEM encoding of the key.

If \arg{pbe} is blank, then the default algorithm (controlled by the
``base/default\_pbe'' option) will be used. As shipped, this default is
``PBE-PKCS5v20(SHA-1,TripleDES/CBC)'' . This is among the more secure options
of PKCS \#5, and is widely supported among implementations of PKCS \#5 v2.0. It
offers 168 bits of security against attacks, which should be more that
sufficient. If you need compatibility with systems that only support PKCS \#5
v1.5, pass ``PBE-PKCS5v15(MD5,DES/CBC)'' as \arg{pbe}. However, be warned that
this PBE algorithm only has 56 bits of security against brute force attacks. As
of 1.4.5, all three keylengths of AES are also available as options, which can
be used with by specifying a PBE algorithm of
``PBE-PKCS5v20(SHA-1,AES-256/CBC)'' (or ``AES-128'' or ``AES-192''). Support
for AES is slightly non-standard, and some applications or libraries might not
handle it. It is known that OpenSSL (0.9.7 and later) do handle AES for private
key encryption.

There may be some strange programs out there that support the v2.0 extensions
to PBES1 but not PBES2; if you need to inter-operate with a program like that,
use ``PBE-PKCS5v15(MD5,RC2/CBC)''. For example, OpenSSL supports this format
(though since it also supports the v2.0 schemes, there is no reason not to just
use TripleDES or AES). This scheme uses a 64 bit key, which, while
significantly better than a 56 bit key, is a bit too small for comfort.

Last but not least, there are some functions which is basically identical to
\function{X509::load\_key}, which will load, and possibly decrypt, a PKCS \#8
private key:

\begin{verbatim}
namespace PKCS8 {
   PKCS8_PrivateKey* load_key(DataSource& in, const User_Interface& ui);
   PKCS8_PrivateKey* load_key(DataSource& in, std::string passphrase = "");

   PKCS8_PrivateKey* load_key(const std::string& filename,
                              const User_Interface& ui);
   PKCS8_PrivateKey* load_key(const std::string& filename,
                              const std::string& passphrase = "");
}
\end{verbatim}

The versions that take \type{std::string} \arg{passphrase}s are primarily for
compatibility, but they are useful in limited circumstances. The
\type{User\_Interface} versions are how \function{load\_key} is actually
implemented, and provides for much more flexibility. Essentially, if the
passphrase given to the function is not correct, then an exception is thrown
and that is that. However, if you pass in an UI object instead, then the UI
object can keep asking the user for the passphrase until they get it right (or
until they cancel the action, though the UI interface). A
\type{User\_Interface} has very little to do with talking to users; it's just a
way to glue together Botan and whatever user interface you happen to be
using. You can think of it as a user interface interface. The default
\type{User\_Interface} is actually very dumb, and effectively acts just like
the versions taking the \type{std::string}.

After loading a key, you can use \function{dynamic\_cast} to find out what
operations it supports, and use it appropriately. Remember to \function{delete}
it once you are done with it.

\subsubsection{Limitations}

As of now Nyberg-Rueppel and Rabin-Williams keys cannot be imported or
exported, because they have no official ASN.1 OID or definition. ElGamal keys
can (as of Botan 1.3.8) be imported and exported, but the only other
implementation which supports the format is Peter Gutmann's Cryptlib. If you
can help it, stick to RSA and DSA.

\emph{Note}: Currently NR and RW are given basic ASN.1 key formats (which
mirror DSA and RSA, respectively), which means that, if they are assigned an
OID, they can be imported and exported just as easily as RSA and DSA. You can
assign them an OID by putting a line in a Botan configuration file, calling
\function{OIDS::add\_oid}, or editing \filename{src/policy.cpp}. Be warned that
it is possible that a future version will use a format which is different from
the current one (\ie, a newly standardized format).

\pagebreak
\section{Certificate Handling}

A certificate is essentially a binding between some identifying information of
a person or other entity (called a \emph{subject}) and a public key. This
binding is asserted by a signature on the certificate, which is placed there by
some authority (the \emph{issuer}) which at least claims that it knows the
subject that is named in the certificate really ``owns'' the private key
corresponding to the public key in the certificate.

The major certificate format in use today is X.509v3, designed by ISO and
further hacked on by dozens (hundreds?) of other organizations.

When working with certificates, the main class to remember is
\type{X509\_Certificate}. You can read an object of this type, but you can't
create one on the fly; a CA object is necessary for actually making a new
certificate. So for the most part, you only have to worry about reading them
in, verifying the signatures, and getting the bits of data in them (most
commonly the public key, and the information about the user of that key). An
X.509v3 certificate can contain a literally infinite number of items related to
all kinds of things. Botan doesn't support a lot of them, simply because nobody
uses them and they're an impossible mess to work with. This section only
documents the most commonly used ones of the ones that are supported; for the
rest, read \filename{x509cert.h} and \filename{asn1\_obj.h} (which has the
definitions of various common ASN.1 constructs used in X.509).

\subsection{So what's in an X.509 certificate?}

Obviously, you want to be able to get the public key. This is achieved by
calling the member function \function{subject\_public\_key}, which will return
a \type{X509\_PublicKey*}. As to what to do with this, read about
\function{load\_key} in the section ``Importing and Exporting PK Keys''. In the
general case, this could be any kind of public key, though 99\% of the time it
will be an RSA key. However, Diffie-Hellman and DSA keys are also supported, so
be careful about how you treat this. It is also a wise idea to examine the
value returned by \function{constraints}, to see what uses the public key is
approved for.

The second major piece of information you'll want is the name/email/etc of the
person to whom this certificate is assigned. Here is where things get a little
nasty. X.509v3 has two (well, mostly just two $\ldots$) different places where
you can stick information about the user: the \emph{subject} field, and in an
extension called \emph{subjectAlternativeName}. The \emph{subject} field is
supposed to only included the following information: country, organization
(possibly), an organizational sub-unit name (possibly), and a so-called common
name. The common name is usually the name of the person, or it could be a title
associated with a position of some sort in the organization. It may also
include fields for state/province and locality. What exactly a locality is,
nobody knows, but it's usually given as a city name.

Botan doesn't currently support any of the Unicode variants used in ASN.1
(UTF-8, UCS-2, and UCS-4), any of which could be used for the fields in the
DN. This could be problematic, particularly in Asia and other areas where
non-ASCII characters are needed for most names. The UTF-8 and UCS-2 string
types \emph{are} accepted (in fact, UTF-8 is used when encoding much of the
time), but if any of the characters included in the string are not in ISO
8859-1 (\ie 0 \ldots 255), an exception will get thrown. Currently the
\type{ASN1\_String} type holds its data as ISO 8859-1 internally (regardless
of local character set); this would have to be changed to hold UCS-2 or UCS-4
in order to support Unicode (also, many interfaces in the X.509 code would have
to accept or return a \type{std::wstring} instead of a \type{std::string}).

Like the distinguished names, subject alternative names can contain a lot of
things that Botan will flat out ignore (most of which you would never actually
want to use). However, there are three very useful pieces of information which
this extension might hold: an email address (``person@site1.com''), a DNS name
(``somehost.site2.com''), or a URI (``http://www.site3.com'').

So, how to get the information? Simply call \function{subject\_info} with the
name of the piece of information you want, and it will return a
\type{std::string} which is either empty (signifying that the certificate
doesn't have this information), or has the information requested. There are
several names for each possible item, but the most easily readable ones are:
``Name'', ``Country'', ``Organization'', ``Organizational Unit'', ``Locality'',
``State'', ``RFC822'', ``URI'', and ``DNS''. These values are returned as a
\type{std::string}.

You can also get information about the issuer of the certificate in the same
way, using \function{issuer\_info}.

\subsubsection{X.509v3 Extensions}

X.509v3 specifies a large number of possible extensions. Botan supports some,
but by no means all of them. This section lists which ones are supported, and
notes areas where there may be problems with the handling. You have to be
pretty familiar with X.509 in order to understand what this is talking about.

\begin{list}{$\cdot$}
  \item Key Usage and Extended Key Usage: No problems known.
  \item

  \item Basic Constraints: No problems known. The default for a v1/v2
        certificate is assume it's a CA if and only if the option
        ``x509/default\_to\_ca'' is set. A v3 certificate is marked as a CA if
        (and only if) the basic constraints extension is present and set for a
        CA cert.

  \item Subject Alternative Names: Only the ``rfc822Name'', ``dNSName'', and
        ``uniformResourceIdentifier'' fields will be stored; all others are
        ignored.

  \item Issuer Alternative Names: Same restrictions as the Subject Alternative
        Names extension. New certificates generated by Botan never include the
        issuer alternative name.

  \item Authority Key Identifier: Only the version using KeyIdentifier is
        supported. If the GeneralNames version is used and the extension is
        critical, an exception is thrown. If both the KeyIdentifier and
        GeneralNames versions are present, then the KeyIdentifier will be
        used, and the GeneralNames ignored.

  \item Subject Key Identifier: No problems known.
\end{list}

\subsubsection{Revocation Lists}

It will occasionally happen that a certificate must be revoked before its
expiration date. Examples of this happening include the private key being
compromised, or the user to which it has been assigned leaving an
organization. Certificate revocation lists are an answer to this problem
(though online certificate validation techniques are starting to become
somewhat more popular). Essentially, every once in a while the CA will release
a CRL, listing all certificates which have been revoked. Also included is
various pieces of information like what time a particular certificate was
revoked, and for what reason. In most systems, it is wise to support some form
of certificate revocation, and CRLs handle this fairly easily.

For most users, processing a CRL is quite easy. All you have to do is call the
constructor, which will take a filename (or a \type{DataSource\&}). The CRLs
can either be in raw BER/DER, or in PEM format; the constructor will figure out
which format without any extra information. For example:

\begin{verbatim}
   X509_CRL crl1("crl1.der");

   DataSource_Stream in("crl2.pem");
   X509_CRL crl2(in);
\end{verbatim}

After that, pass the \type{X509\_CRL} object to a \type{X509\_Store} object
with \type{X509\_Code} \function{add\_crl}(\type{X509\_CRL}), and all future
verifications will take into account the certificates listed, assuming
\function{add\_crl} returns \type{VERIFIED}. If it doesn't return
\type{VERIFIED}, then the return value is an error code signifying that the CRL
could not be processed due to some problem (which could range from the issuing
certificate not being found, to the CRL having some format problem). For more
about the \type{X509\_Store} API, read the section later in this chapter.

\subsection{Reading Certificates}

\type{X509\_Certificate} has two constructors, each of which takes a source of
data; a filename to read, and a \type{DataSource\&}.

\subsection{Storing and Using Certificates}

If you read a certificate, you probably want to verify the signature on
it. However, consider that to do so, we may have to verify the signature on the
certificate that we used to verify the first certificate, and on and on until
we hit the top of the certificate tree somewhere. It would be a might huge pain
to have to handle all of that manually in every application, so there is
something that does it for you: \type{X509\_Store}.

This is a pretty easy thing to use. The basic operations are: put certificates
and CRLs into it, search for certificates, and attempt to verify
certificates. That's about it. In the future, there will be support for online
retrieval of certificates and CRLs (\eg with the HTTP cert-store interface
currently under consideration by PKIX).

\subsubsection{Adding Certificates}

You can add new certificates to a certificate store using any of these
functions:

\function{add\_cert}(\type{const X509\_Certificate\&} \arg{cert},
                     \type{bool} \arg{trusted} \type{= false})

\function{add\_certs}(\type{DataSource\&} \arg{source})

\function{add\_trusted\_certs}(\type{DataSource\&} \arg{source})

The versions that take a \type{DataSource\&} will add all of the certificates
that it can find in that source.

All of them add the cert(s) to the store. The 'trusted' certificates are the
ones which you have some reason to trust are genuine. For example, say your
application is working with certificates which are owned by employees of some
company, and all of their certificates are signed by the company CA, whose
certificate is in turned signed by a commercial root CA. What you would then do
is include the certificate of the commercial CA with your application, and read
it in as a trusted certificate. From there, you could verify the company CA's
certificate, and then use that to verify the end user's certificates. Only
self-signed certificates may be considered trusted.

\subsubsection{Adding CRLs}

\type{X509\_Code} \function{add\_crl}(\type{const X509\_CRL\&} \arg{crl});

This will process the CRL and mark the revoked certificates. This will also
work if a revoked certificate is added to the store sometime after the CRL is
processed. The function can return an error code (listed later), or will return
\type{VERIFIED} if everything completed successfully.

\subsubsection{Storing Certificates}

You can output a set of certificates by calling \function{PEM\_encode}, which
will return a \type{std::string} containing each of the certificates in the
store, PEM encoded and concatenated. This simple format can easily be read by
both Botan and other libraries/applications.

\subsubsection{Searching for Certificates}

You can find certificates in the store with a series of functions contained
in the \function{X509\_Store\_Search} namespace:

\begin{verbatim}
namespace X509_Store_Search {
std::vector<X509_Certificate> by_email(const X509_Store& store,
                                       const std::string& email_addr);
std::vector<X509_Certificate> by_name(const X509_Store& store,
                                      const std::string& name);
std::vector<X509_Certificate> by_dns(const X509_Store&,
                                     const std::string& dns_name);
}
\end{verbatim}

These functions will return a (possibly empty) vector of certificates from
\arg{store} matching your search criteria. The email address and DNS name
searches are case-insensitive but are sensitive to extra whitespace and so
on. The name search will do case-insensitive substring matching, so, for
example, calling \function{X509\_Store\_Search::by\_name}(\arg{your\_store},
``dob'') will return certificates for ``J.R. 'Bob' Dobbs'' and
``H. Dobbertin'', assuming both of those certificates are in \arg{your\_store}.

You could then display the results to a user, and allow them to select the
appropriate one. Searching using an email address as the key is usually more
effective than the name, since email addresses are rarely shared.

\subsubsection{Certificate Stores}

An object of type \type{Certificate\_Store} is a generalized interface to an
external source for certificates (and CRLs). Examples of such a store would be
one that looked up the certificates in a SQL database, or by contacting a CGI
script running on a HTTP server. There are currently three mechanisms for
looking up a certificate, and one for retrieving CRLs. By default, most of
these mechanisms will simply return an empty \type{std::vector} of
\type{X509\_Certificate}. This storage mechanism is \emph{only} queried when
doing certificate validation: it allows you to distribute only the root key
with an application, and let some online method handle getting all the other
certificates that are needed to validate an end entity certificate. In
particular, the search routines will not attempt to access the external
database.

The three certificate lookup methods are \function{by\_SKID} (Subject Key
Identifier), \function{by\_name} (the CommonName DN entry), and
\function{by\_email} (stored in either the distinguished name, or in a
subjectAlternativeName extension). The name and email versions take a
\type{std::string}, while the SKID version takes a \type{SecureVector<byte>}
containing the subject key identifier in raw binary. You can choose not to
implement \function{by\_name} or \function{by\_email}, but \function{by\_SKID}
is mandatory to implement, and, currently, is the only version which is used by
\type{X509\_Store}.

Finally, there is a method for finding CRLs, called \function{get\_crls\_for},
which takes an \type{X509\_Certificate} object, and returns a
\type{std::vector} of \type{X509\_CRL}. While generally there will be only one
CRL, the use of the vector makes it easy to return no CRLs (\eg, if the
certificate store doesn't support retrieving them), or return multiple ones
(for example, if the certificate store can't determine precisely which key was
used to sign the certificate). Implementing the function is optional, and by
default will return no CRLs. If it is available, it will be used by
\type{X509\_CRL}.

As for actually using such a store, you have to tell \type{X509\_Store} about
it, by calling the \type{X509\_Store} member function

\function{add\_new\_certstore}(\type{Certificate\_Store}* \arg{new\_store})

The argument, \arg{new\_store}, will be deleted by \type{X509\_Store}'s
destructor, so make sure to allocate it with \function{new}.

\subsubsection{Verifying Certificates}

There is a single function in \type{X509\_Store} related to verifying a
certificate:

\type{X509\_Code}
\function{validate\_cert}(\type{const X509\_Certificate\&} \arg{cert},
                          \type{Cert\_Usage} \arg{usage} = \type{ANY})

To sum things up simply, it returns \type{VERIFIED} if the certificate can
safely be considered valid for the usage(s) described by \arg{usage}, and an
error code if it is not. Naturally, things are a bit more complicated than
that. The enum \type{Cert\_Usage} is defined inside the \type{X509\_Store}
class, it (currently) can take on any of the values \type{ANY} (any usage is
OK), \type{TLS\_SERVER} (for SSL/TLS server authentication), \type{TLS\_CLIENT}
(for SSL/TLS client authentication), \type{CODE\_SIGNING},
\type{EMAIL\_PROTECTION} (email encryption, usually this means S/MIME),
\type{TIME\_STAMPING} (in theory any time stamp application, usually IETF
PKIX's Time Stamp Protocol), or \type{CRL\_SIGNING}. Note that Microsoft's code
signing system, certainly the most widely used, uses a completely different
(and basically undocumented) method for marking certificates for code signing.

First, how does it know if a certificate is valid? Basically, a certificate is
valid if both of the following hold: a) the signature in the certificate can be
verified using the public key in the issuer's certificate, and b) the issuer's
certificate is a valid CA certificate. Note that this definition is
recursive. We get out of this by ``bottoming out'' when we reach a certificate
that we consider trusted. In general this will either be a commercial root CA,
or an organization or application specific CA.

There are actually a few other restrictions (validity periods, key usage
restrictions, etc), but the above summarizes the major points of the validation
algorithm. In theory, Botan implements the certificate path validation
algorithm given in RFC 2459, but in practice it does not (yet), because we
don't support the X.509v3 policy or name constraint extensions.

Possible values for \arg{usage} are \type{TLS\_SERVER}, \type{TLS\_CLIENT},
\type{CODE\_SIGNING}, \type{EMAIL\_PROTECTION}, \type{CRL\_SIGNING}, and
\type{TIME\_STAMPING}, and \type{ANY}. The default \type{ANY} does not mean
valid for any use, it means ``is valid for some usage''. This is generally
fine, and in fact requiring that a random certificate support a particular
usage will likely result in a lot of failures, unless your application is very
careful to always issue certificates with the proper extensions, and you never
use certificates generated by other apps.

Return values for \function{validate\_cert} (and \function{add\_crl}) include:

\begin{list}{$\cdot$}
  \item VERIFIED: The certificate is valid for the specified use.
  \item
  \item INVALID\_USAGE: The certificate cannot be used for the specified use.

   \item CANNOT\_ESTABLISH\_TRUST: The root certificate was not marked as
         trusted.
   \item CERT\_CHAIN\_TOO\_LONG: The certificate chain exceeded the length
         allowed by a basicConstraints extension.
   \item SIGNATURE\_ERROR: An invalid signature was found
   \item POLICY\_ERROR: Some problem with the certificate policies was found.

   \item CERT\_FORMAT\_ERROR: Some format problem was found in a certificate.
   \item CERT\_ISSUER\_NOT\_FOUND: The issuer of a certificate could not be
         found.
   \item CERT\_NOT\_YET\_VALID: The certificate is not yet valid.
   \item CERT\_HAS\_EXPIRED: The certificate has expired.
   \item CERT\_IS\_REVOKED: The certificate has been revoked.

   \item CRL\_FORMAT\_ERROR: Some format problem was found in a CRL.
   \item CRL\_ISSUER\_NOT\_FOUND: The issuer of a CRL could not be found.
   \item CRL\_NOT\_YET\_VALID: The CRL is not yet valid.
   \item CRL\_HAS\_EXPIRED: The CRL has expired.

   \item CA\_CERT\_CANNOT\_SIGN: The CA certificate found does not have an
         contain a public key that allows signature verification.
   \item CA\_CERT\_NOT\_FOR\_CERT\_ISSUER: The CA cert found is not allowed to
         issue certificates.
   \item CA\_CERT\_NOT\_FOR\_CRL\_ISSUER: The CA cert found is not allowed to
         issue CRLs.

  \item UNKNOWN\_X509\_ERROR: Some other error occurred.

\end{list}

\subsection{Certificate Authorities}

Setting up a CA for X.509 certificates is actually probably the easiest thing
to do related to X.509. A CA is represented by the type \type{X509\_CA}, which
can be found in \filename{x509\_ca.h}. A CA always needs its own certificate,
which can either be a self-signed certificate (see below on how to create one)
or one issued by another CA (see the section on PKCS \#10 requests). Creating
a CA object is done by the following constructor:

\begin{verbatim}
   X509_CA(const X509_Certificate& cert, const PKCS8_PrivateKey& key);
\end{verbatim}

The private key is the private key corresponding to the public key in the the
CA's certificate.

Generally, requests for new certificates are supplied to a CA in the form on
PKCS \#10 certificate requests (called a \type{PKCS10\_Request} object in
Botan). These are decoded in a similar manner to
certificates/CRLs/etc. Generally, a request is vetted by humans (who somehow
verify that the name in the request corresponds to the name of the person who
requested it), and then signed by a CA key, generating a new certificate.

\begin{verbatim}
   X509_Certificate sign_request(const PKCS10_Request&) const;
\end{verbatim}

\subsubsection{Generating CRLs}

As mentioned previously, the ability to process CRLs is highly important in
many PKI systems. In fact, according to strict X.509 rules, you must not
validate any certificate if the appropriate CRLs are not available (though
hardly any systems are that strict). In any case, a CA should have a valid CRL
available at all times.

Of course, you might be wondering what to do if no certificates have been
revoked. In fact, CRLs can be issued without any actually revoked certificates
- the list of certs will simply be empty. To generate a new, empty CRL, just
call \type{X509\_CRL}
\function{X509\_CA::new\_crl}(\type{u32bit}~\arg{seconds}~=~0)~--~it will
create a new, empty, CRL. If \arg{seconds} is the default 0, then the normal
default CRL next update time (the value of the ``x509/crl/next\_update'') will
be used. If not, then \arg{seconds} specifies how long (in seconds) it will be
until the CRL's next update time (after this time, most clients will reject the
CRL as too old).

On the other hand, you may have issued a CRL before. In which case, you will
want to issue a new CRL which contains both all previously revoked
certificates, along with any new ones. This is done by calling the
\type{X509\_CA} member function
\function{update\_crl}(\type{X509\_CRL}~\arg{old\_crl},
\type{std::vector<CRL\_Entry>}~\arg{new\_revoked},
\type{u32bit}~\arg{seconds}~=~0), where \type{X509\_CRL} is the last CRL this
CA issued, and \arg{new\_revoked} is a list of any newly revoked certificates.
The function returns a new \type{X509\_CRL} to make available for clients. The
semantics for the \arg{seconds} argument is the same as \function{new\_crl}.

The \type{CRL\_Entry} type is a structure which contains, at a minimum, the
serial number of the revoked certificate. As serial numbers are never repeated,
the pairing of an issuer and a serial number (should) distinctly identify any
certificate. In this case, we represent the serial number as a
\type{SecureVector<byte>} called \arg{serial}. There are two additional
(optional) values, an enumeration called \type{CRL\_Code} which specifies the
reason for revocation (\arg{reason}), and an object which represents the time
that the certificate became invalid (if this information is known).

If you wish to remove an old entry from the CRL, insert a new entry for the
same cert, with a \arg{reason} code of \type{DELETE\_CRL\_ENTRY}. For example,
if a revoked certificate has expired 'normally', there is no reason to continue
to explicitly revoke it, since clients will reject the cert as expired in any
case.

\subsubsection{Self-Signed Certificates}

Generating a new self-signed certificate can often be useful, for example when
setting up a new root CA, or for use in email applications. In this case,
the solution is summed up simply as:

\begin{verbatim}
namespace X509 {
   X509_Certificate create_self_signed_cert(const X509_Cert_Options& opts,
                                            const PKCS8_PrivateKey& key);
}
\end{verbatim}

Where \arg{key} is obviously the private key you wish to use (the public key,
used in the certificate itself, is extracted from the private key), and
\arg{opts} is an structure which has various bits of information which will be
used in creating the certificate (this structure, and its use, is discussed
below). This function is found in the header \filename{x509self.h}. There is an
example of using this function in the \filename{self\_sig} example.

\subsubsection{Creating PKCS \#10 Requests}

Also in \filename{x509self.h}, there is a function for generating new PKCS \#10
certificate requests.

\begin{verbatim}
namespace X509 {
   PKCS10_Request create_cert_req(const X509_Cert_Options&,
                                  const PKCS8_PrivateKey&);
}
\end{verbatim}

This function acts quite similarly to \function{create\_self\_signed\_cert},
except it instead returns a PKCS \#10 certificate request. After creating it,
one would typically transmit it to a CA, who signs it and returns a freshly
minted X.509 certificate. There is an example of using this function in the
\filename{pkcs10} example.

\subsubsection{Certificate Options}

So what is this \type{X509\_Cert\_Options} thing we've been passing around?
Basically, it's a bunch of information which will end up being stored into the
certificate. This information comes in 3 major flavors: information about the
subject (CA or end-user), the validity period of the certificate, and
restrictions on the usage of the certificate.

First and foremost is a number of \type{std::string} members, which contains
various bits of information about the user: \arg{common\_name},
\arg{serial\_number}, \arg{country}, \arg{organization}, \arg{org\_unit},
\arg{locality}, \arg{state}, \arg{email}, \arg{dns\_name}, and \arg{uri}. As
many of these as possible should be filled it (especially an email address),
though the only required ones are \arg{common\_name} and \arg{country}.

There is another value which is only useful when creating a PKCS \#10 request,
which is called \arg{challenge}. This is a challenge password, which you can
later use to request certificate revocation (\emph{if} the CA supports doing
revocations in this manner).

Then there is the validity period; these are set with \function{not\_before}
and \function{not\_after}. Both of these functions also take a
\type{std::string}, which specifies when the certificate should start being
valid, and when it should stop being valid. If you don't set the starting
validity period, it will automatically choose the current time. If you don't
set the ending time, it will choose the starting time plus a default time
period. The arguments to these functions specify the time in the following
format: ``2002/11/27 1:50:14''. The time is in 24 hour format, and the date is
encoded as year/month/day. The date must be specified, but you can omit the
time or trailing parts of it, for example ``2002/11/27 1:50'' or
``2002/11/27''.

Lastly, you can set constraints on a key. The one you're mostly likely to want
to use is to create (or request) a CA certificate, which can be done by calling
the member function \function{CA\_key}. This should only be used when needed.

Other constraints can be set by calling the member functions
\function{add\_constraints} and \function{add\_ex\_constraints}. The first takes
a \type{Key\_Constraints} value, and replaces any previously set value. If no
value is set, then the certificate key is marked as being valid for any usage.
You can set it to any of the following (for more than one usage, OR them
together): \type{DIGITAL\_SIGNATURE}, \type{NON\_REPUDIATION},
\type{KEY\_ENCIPHERMENT}, \type{DATA\_ENCIPHERMENT}, \type{KEY\_AGREEMENT},
\type{KEY\_CERT\_SIGN}, \type{CRL\_SIGN}, \type{ENCIPHER\_ONLY},
\type{DECIPHER\_ONLY}. Many of these have quite special semantics, so you
should either consult the appropriate standards document (such as RFC 3280), or
simply not call \function{add\_constraints}, in which case the appropriate
values will be chosen for you.

The second function, \function{add\_ex\_constraints}, allows you to specify an
OID which has some meaning with regards to restricting the key to particular
usages. You can, if you wish, specify any OID you like, but there are a set of
standard ones which other applications will be able to understand. These are
the ones specified by the PKIX standard, and are named ``PKIX.ServerAuth'' (for
TLS server authentication), ``PKIX.ClientAuth'' (for TLS client
authentication), ``PKIX.CodeSigning'', ``PKIX.EmailProtection'' (most likely
for use with S/MIME), ``PKIX.IPsecUser'', ``PKIX.IPsecTunnel'',
``PKIX.IPsecEndSystem'', and ``PKIX.TimeStamping''. You can call
\function{add\_ex\_constraints} any number of times~--~each new OID will be
added to the list to include in the certificate.

\pagebreak
\section{The Low-Level Interface}

Botan has two different interfaces. The one documented in this section is meant
more for implementing higher-level types (see the section on filters, later in
this manual) than for use by applications. Using it safely requires a solid
knowledge of encryption techniques and best practices, so unless you know, for
example, what CBC mode and nonces are, and why PKCS \#1 padding is important,
you should avoid this interface in favor of something working at a higher level
(such as the CMS interface).

\subsection{Basic Algorithm Abilities}

There are a small handful of functions implemented by most of Botan's
algorithm objects. Among these are:

\noindent
\type{std::string} \function{name}():

Returns a human-readable string of the name of this algorithm. Examples of
names returned are ``Blowfish'' and ``HMAC(MD5)''. You can turn names back into
algorithm objects using the functions in \filename{lookup.h}.

\noindent
\type{void} \function{clear}():

Clear out the algorithm's internal state. A block cipher object will ``forget''
its key, a hash function will ``forget'' any data put into it, etc. Basically,
the object will look exactly as it did when you initially allocated it.

\noindent
\function{clone}():

This function is central to Botan's name-based interface. The \function{clone}
has many different return types, such as \type{BlockCipher*} and
\type{HashFunction*}, depending on what kind of object it is called on. Note
that unlike Java's clone, this returns a new object in a ``pristine'' state;
that is, operations done on the initial object before calling \function{clone}
do not affect the initial state of the new clone.

Cloned objects can (and should) be deallocated with the C++ \texttt{delete}
operator.

\subsection{Keys and IVs}

Both symmetric keys and initialization values can simply be considered byte (or
octet) strings. These are represented by the classes \type{SymmetricKey} and
\type{InitializationVector}, which are subclasses of \type{OctetString}.

Since often it's hard to distinguish between a key and IV, many things (such as
key derivation mechanisms) return \type{OctetString} instead of
\type{SymmetricKey} to allow its use as a key or an IV.

\noindent
\function{OctetString}(\type{u32bit} \arg{length}):

This constructor creates a new random key of size \arg{length}.

\noindent
\function{OctetString}(\type{std::string} \arg{str}):

The argument \arg{str} is assumed to be a hex string; it is converted to binary
and stored. Whitespace is ignored.

\noindent
\function{OctetString}(\type{const byte} \arg{input}[], \type{u32bit}
\arg{length}):

This constructor simply copies its input.

\subsection{Symmetrically Keyed Algorithms}

Block ciphers, stream ciphers, and MACs all handle keys in pretty much the same
way. To make this similarity explicit, all algorithms of those types are
derived from the \type{SymmetricAlgorithm} base class. This type has three
functions:

\noindent
\type{void} \function{set\_key}(\type{const byte} \arg{key}[], \type{u32bit}
\arg{length}):

Most algorithms only accept keys of certain lengths. If you attempt to call
\function{set\_key} with a key length that is not supported, the exception
\type{Invalid\_Key\_Length} will be thrown. There is also another version of
\function{set\_key} that takes a \type{SymmetricKey} as an argument.

\noindent
\type{bool} \function{valid\_keylength}(\type{u32bit} \arg{length}) const:

This function returns true if a key of the given length will be accepted by
the cipher.

There are also three constant data members of every \type{SymmetricAlgorithm}
object, which specify exactly what limits there are on keys which that object
can accept:

MAXIMUM\_KEYLENGTH: The maximum length of a key. Usually, this is at most 32
(256 bits), even if the algorithm actually supports more. In a few rare cases
larger keys will be supported.

MINIMUM\_KEYLENGTH: The minimum length of a key. This is at least 1.

KEYLENGTH\_MULTIPLE: The length of the key must be a multiple of this value.

In all cases, \function{set\_key} must be called on an object before any data
processing (encryption, decryption, etc) is done by that object. If this is not
done, the results are undefined -- that is to say, Botan reserves the right in
this situation to do anything from printing a nasty, insulting message on the
screen to dumping core.

\subsection{Block Ciphers}

Block ciphers implement the interface \type{BlockCipher}, found in
\filename{base.h}, as well as the \type{SymmetricAlgorithm} interface.

\noindent
\type{void} \function{encrypt}(\type{const byte} \arg{in}[BLOCK\_SIZE],
                               \type{byte} \arg{out}[BLOCK\_SIZE]) const

\noindent
\type{void} \function{encrypt}(\type{byte} \arg{block}[BLOCK\_SIZE]) const

These functions apply the block cipher transformation to \arg{in} and
place the result in \arg{out}, or encrypts \arg{block} in place
(\arg{in} may be the same as \arg{out}). BLOCK\_SIZE is a constant
member of each class, which specifies how much data a block cipher can
process at one time. Note that BLOCK\_SIZE is not a static class
member, meaning you can (given a \type{BlockCipher*} named
\arg{cipher}), call \verb|cipher->BLOCK_SIZE| to get the block size of
that particular object. \type{BlockCipher}s have similar functions
\function{decrypt}, which perform the inverse operation.

\begin{verbatim}
AES_128 cipher;
SymmetricKey key(cipher.MAXIMUM_KEYLENGTH); // randomly created
cipher.set_key(key);

byte in[16] = { /* secrets */ };
byte out[16];
cipher.encrypt(in, out);
\end{verbatim}

\subsection{Stream Ciphers}

Stream ciphers are somewhat different from block ciphers, in that encrypting
data results in changing the internal state of the cipher. Also, you may
encrypt any length of data in one go (in byte amounts).

\noindent
\type{void} \function{encrypt}(\type{const byte} \arg{in}[], \type{byte}
\arg{out}[], \type{u32bit} \arg{length})

\noindent
\type{void} \function{encrypt}(\type{byte} \arg{data}[], \type{u32bit}
\arg{length}):

These functions encrypt the arbitrary length (well, less than 4 gigabyte long)
string \arg{in} and place it into \arg{out}, or encrypts it in place in
\arg{data}. The \function{decrypt} functions look just like
\function{encrypt}.

Stream ciphers implement the \type{SymmetricAlgorithm} interface.

Some stream ciphers support random access to any point in their cipher
stream. For such ciphers, calling \type{void} \function{seek}(\type{u32bit}
\arg{byte}) will change the cipher's state so that it as if the cipher had been
keyed as normal, then encrypted \arg{byte} -- 1 bytes of data (so the next byte
in the cipher stream is byte number \arg{byte}).

\subsection{Hash Functions / Message Authentication Codes}

Hash functions take their input without producing any output, only producing
anything when all input has already taken place. MACs are very similar, but are
additionally keyed. Both of these are derived from the base class
\type{BufferedComputation}, which has the following functions.

\noindent
\type{void} \function{update}(\type{const byte} \arg{input}[], \type{u32bit}
\arg{length})

\noindent
\type{void} \function{update}(\type{byte} \arg{input})

\noindent
\type{void} \function{update}(\type{const std::string \&} \arg{input})

Updates the hash/mac calculation with \arg{input}.

\noindent
\type{void} \function{final}(\type{byte} \arg{out}[OUTPUT\_LENGTH])

\noindent
\type{SecureVector<byte>} \function{final}():

Complete the hash/MAC calculation and place the result into \arg{out}.
OUTPUT\_LENGTH is a public constant in each object that gives the length of the
hash in bytes. After you call \function{final}, the hash function is reset to
its initial state, so it may be reused immediately.

The second method of using final is to call it with no arguments at all, as
shown in the second prototype. It will return the hash/mac value in a memory
buffer, which will have size OUTPUT\_LENGTH.

There are also a pair of functions called \function{process}. They are
essentially a combination of a single \function{update}, and \function{final}.
Both versions return the final value, rather than placing it an array. Calling
\function{process} with a single byte value isn't available, mostly because it
would rarely be useful.

A MAC can be viewed (in most cases) as simply a keyed hash function, so classes
which are derived from \type{MessageAuthenticationCode} have \function{update}
and \function{final} classes just like a \type{HashFunction} (and like a
\type{HashFunction}, after \function{final} is called, it can be used to make a
new MAC right away; the key is kept around).

A MAC has the \type{SymmetricAlgorithm} interface in addition to the
\type{BufferedComputation} interface.

\pagebreak
\section{Random Number Generators}

The random number generators provided in Botan are meant for creating keys,
IVs, padding, nonces, and anything else which requires 'random' data. It is
important to remember that the output of these classes will vary, even if they
are supplied with exactly the same seed (\ie, two \type{Randpool} objects with
similar initial states will not produce the same output, because the value of
high resolution timers is added to the state at various points).

To ensure good quality output, a PRNG needs to be seeded with truly random data
(such as that produced by a hardware RNG). Typically, you will use an
\type{EntropySource} (see below). To add entropy to a PRNG, you can use
\type{void} \function{add\_entropy}(\type{const byte} \arg{data}[],
\type{u32bit} \arg{length}) or (better), use the \type{EntropySource}
interface.

One a PRNG has been initialized, you can get a single byte of random data by
calling \type{byte} \function{random()}, or get a large block by calling
\type{void} \function{randomize}(\type{byte} \arg{data}[], \type{u32bit}
\arg{length}), which will put random bytes into each member of the array from
indexes 0 $\ldots$ \arg{length} -- 1.

You can avoid all the problem inherent to seeding the PRNG by using the
globally shared PRNG, described later in this section.

\subsection{Entropy Estimation}

The PRNG algorithms included in Botan have various sanity checks included. In
particular, they try to make sure that a reasonable amount of entropy has been
input into them before they will output any randomness. If this condition is
not met, they will throw a \type{PRNG\_Unseeded} exception. While generally a
library shouldn't be making policy decisions for applications, it seems
generally preferable for the application to fail than for it to generate
insecure keys.

On Windows and Unix systems, the available entropy source modules can provide
more than enough entropy to seed the PRNGs sufficiently. However, if these
entropy sources aren't compiled into the library, the application will have to
handle seeding on its own.

\subsection{The Global PRNG}

Botan maintains a global PRNG (actually, a pair of them) that is used
internally for things like generating secret keys and salts. These PRNGs are
automatically seeded by the \type{LibraryInitializer}. Most of the time, you
won't need to access it directly because the library handles the common cases
where randomness is needed for you, but you might want to for a complicated
application (or when implementing things at a low level).

To use it, include \filename{rng.h}. You can't get a pointer to the actual
global PRNG object, because it is guarded with a mutex for thread safety, so
the interface basically defines a set of entry points into the object. All of
them are in the namespace \namespace{Global\_RNG}, which is inside the
\namespace{Botan} namespace. So you might call them as
\texttt{Botan::Global\_RNG::function}, or if you have a \keyword{using}
declaration to include Botan objects into the global namespace, just
\texttt{Global\_RNG::function}.

There are six functions, four for adding entropy and two for getting
randomness out.

\vskip 5pt
\noindent
\type{void} \function{Global\_RNG::randomize}(\type{byte} \arg{buf[]},
                                              \type{u32bit} \arg{size})

Get \arg{size} bytes of random bytes from the global PRNG and put it into
\arg{buf}.

\vskip 5pt
\noindent
\type{byte} \function{Global\_RNG::random}():

Return a single random byte

\vskip 5pt
\noindent
\type{void} \function{Global\_RNG::add\_entropy}(\type{const byte} \arg{buf}[],
                                                 \type{u32bit} \arg{size}):

Add the contents of \arg{buf}, which is of size \arg{size}, into the global
PRNG's internal state. The contents of the buffer cannot be recovered from the
PRNG output or internal state, and the PRNGs included in Botan are specifically
designed to be safe even if fed large amounts of data chosen by an attacker
trying to weaken the PRNG. So feel free to include things like data you
received over a socket (if you're writing a network application), passwords,
log data, etc.

\vskip 5pt
\noindent
\type{void} \function{Global\_RNG::add\_entropy}(\type{EntropySource\&}
                        \arg{es}, \type{bool} \arg{slow\_poll}):

Poll \arg{es} for entropy. If \arg{slow\_poll} is true, then do a slow poll,
otherwise do a fast poll.

\vskip 5pt
\noindent
\type{u32bit} \function{Global\_RNG::seed}
(\type{bool} \arg{slow\_poll} = \arg{true},
 \type{u32bit} \arg{bits\_to\_get} = 256)

Seed the global PRNG, either a fast or slow poll (default a slow), until it
gets at least \arg{bits\_to\_get} bits of entropy. However, if little entropy
is available on the system, it's entirely possible it will retrieve less than
that (particularly if a fast poll is being done). This function will return an
estimate for how many bits were gathered by the seeding process.

If you pass 0 for \arg{bits\_to\_get}, then a poll will be run from all
available entropy sources. Usually if enough entropy is collected after a few
sources, the function will exit early. This is especially useful if you don't
trust \filename{/dev/urandom} to be safe for some reason.

If you've got a long running server process, it's a good idea to create a
thread that just calls this function every once in a while, sleeping the rest
of the time. Make sure to cancel it before you shutdown the library, though;
otherwise it will try to get memory from the now-nonexist allocators, fail, and
throw an exception (or crash). An alternate method might be to call it after
servicing a particular number of clients.

\vskip 5pt
\noindent
\type{u32bit} \function{Global\_RNG::add\_es}
(\type{EntropySource*} \arg{source}, \type{bool} \arg{last} = \arg{true})

Normally the library generates a list of entropy sources for
\function{Global\_RNG::seed} to call at initialization time. With this function
you can add new entropy sources which will be queried. If \arg{last} is true,
the the entropy source is put at the end of the list of currently used entropy
sources. If you'd like to be sure that your source is always called, set
\arg{last} to \arg{false}, in which case it will placed at the start of the
list.

\subsection{Randpool}

\type{Randpool} is the primary PRNG within Botan. In recent versions all uses
of it have been wrapped by an implementation of the X9.31 PRNG (see below). If
for some reason you should have cause to create a PRNG instead of using the
``global'' one owned by the library, it would be wise to consider the same on
the grounds of general caution; while \type{Randpool} is designed with known
attacks and PRNG weaknesses in mind, it is not an standard/official PRNG. The
remainer of this section is a (fairly technical, though high-level) description
of the algorithms used in this PRNG. Unless you have a specific interest in
this subject, the rest of this section might prove somewhat uninteresting.

\type{Randpool} has an internal state called pool, which is 512 bytes
long. This is where entropy is mixed into and extracted from. There is also a
small output buffer (called buffer), which holds the data which has already
been generated but has just not been output yet.

It is based around a MAC and a block cipher (which are currently HMAC(SHA-256)
and AES-256). Where a specific size is mentioned, it should be taken as a
multiple of the cipher's block size. For example, if a 256-bit block cipher
were used instead of AES, all of the sizes internally would double. Every time
some new output is needed, we compute the MAC of a counter and a high
resolution timer. The resulting MAC is XORed into the output buffer (wrapping
as needed), and the output buffer is then encrypted with AES, producing 16
bytes of output.

After 8 blocks (or 128 bytes) have been produced, we mix the pool. To do this,
we first rekey both the MAC and the cipher; the new MAC key is the MAC of the
current pool under the old MAC key, while the new cipher key is the MAC of the
current pool under the just-chosen MAC key. We then encrypt the entire pool in
CBC mode, using the current (unused) output buffer as the IV. We then generate
a new output buffer, using the mechanism described in the previous paragraph.

To add randomness to the PRNG, we compute the MAC of the input and XOR the
output into the start of the pool. Then we remix the pool and produce a new
output buffer. The initial MAC operation should make it very hard for chosen
inputs to harm the security of \type{Randpool}, and as HMAC should be able to
hold roughly 256 bits of state, it is unlikely that we are wasting much input
entropy (or, if we are, it doesn't matter, because we have a very abundant
supply).

\subsection{ANSI X9.31}

\type{ANSI\_X931\_PRNG} is the standard issue X9.31 Appendix A.2.4 PRNG, though
using AES-256 instead of 3DES as the block cipher. This PRNG implementation has
been checked against official X9.31 test vectors.

Internally, the PRNG holds a pointer to another PRNG (typically
Randpool). This internal PRNG generates the key and seed used by the
X9.31 algorithm, as well as the date/time vectors. Each time an X9.31
PRNG object recieves entropy, it simply passes it along to the PRNG it
is holding, and then pulls out some random bits to generate a new key
and seed. This PRNG considers itself seeded as soon as the internal
PRNG is seeded.

As of version 1.4.7, the X9.31 PRNG is by default used for all random number
generation.

\subsection{Entropy Sources}

An \type{EntropySource} is an abstract representation of some method of gather
``real'' entropy. This tends to be very system dependent. The \emph{only} way
you should use an \type{EntropySource} is to pass it to a PRNG that will
extract entropy from it -- never use the output directly for any kind of key or
nonce generation!

\type{EntropySource} has a pair of functions for getting entropy from some
external source, called \function{fast\_poll} and \function{slow\_poll}. These
pass a buffer of bytes to be written; the functions then return how many bytes
of entropy were actually gathered. \type{EntropySource}s are usually used to
seed the global PRNG using the functions found in the \namespace{Global\_RNG}
namespace.

Note for writers of \type{EntropySource}s: it isn't necessary to use any kind
of cryptographic hash on your output. The data produced by an EntropySource is
only used by an application after it has been hashed by the
\type{RandomNumberGenerator} which asked for the entropy, and thus any hashing
you do will be wasteful of both CPU cycles and possibly entropy.

\pagebreak
\section{User Interfaces}

Botan has recently changed some infrastructure to better accommodate more
complex user interfaces, in particular ones which are based on event
loops. Primary among these was the fact that when doing something like loading
a PKCS \#8 encoded private key, a passphrase might be needed, but then again it
might not (a PKCS \#8 key doesn't have to be encrypted). Asking for a
passphrase to decrypt an unencrypted key is rather pointless. Not only that,
but the way to handle the user typing the wrong passphrase was complicated,
undocumented, and inefficient.

So now Botan has an object called \type{UI}, which provides a simple interface
for the aspects of user interaction the library has to be concerned
with. Currently, this means getting a passphrase from the user, and that's it
(\type{UI} will probably be extended in the future to support other operations
as they are needed). The base \type{UI} class is very stupid, because the
library can't directly assume anything about the environment that it's running
under (for example, if there will be someone sitting at the terminal, if the
application is even \emph{attached} to a terminal, and so on). But since you
can subclass \type{UI} to use whatever method happens to be appropriate for
your application, this isn't a big deal.

There is (currently) a single function that can be overridden by subclasses of
\type{UI} (the \type{std::string} arguments are actually \type{const
std::string\&}, but shown as simply \type{std::string} to keep the line from
wrapping):

\noindent
\type{std::string} \function{get\_passphrase}(\type{std::string} \arg{what},
                                              \type{std::string} \arg{source},
                                     \type{UI\_Result\&} \arg{result}) const;

The \arg{what} argument specifies what the passphrase is needed for (for
example, PKCS \#8 key loading passes \arg{what} as ``PKCS \#8 private
key''). This lets you provide the user with some indication of \emph{why} your
application is asking for a passphrase; feel free to pass the string through
\function{gettext(3)} or moral equivalent for i18n purposes. Similarly,
\arg{source} specifies where the data in question came from, if available (for
example, a file name). If the source is not available for whatever reason, then
\arg{source} will be an empty string; be sure to account for this possibility
when writing a \type{UI} subclass.

The function returns the passphrase as the return value, and a status code in
\arg{result} (either \type{OK} or \type{CANCEL\_ACTION}). If
\type{CANCEL\_ACTION} is returned in \arg{result}, then the return value will
be ignored, and the caller will take whatever action is necessary (typically,
throwing an exception stating that the passphrase couldn't be determined). In
the specific case of PKCS \#8 key decryption, a \type{Decoding\_Error}
exception will be thrown; your UI should assume this can happen, and provide
appropriate error handling (such as putting up a dialog box informing the user
of the situation, and canceling the operation in progress).

There is an example \type{UI} which uses GTK+ available on the web site. The
\type{GTK\_UI} code is cleanly separated from the rest of the example, so if
you happen to be using GTK+, you can copy (and/or adapt) that code for your
application. If you write a \type{UI} object for another windowing system
(Win32, Qt, wxWindows, FOX, etc), and would like to make it available to users
in general (ideally under a permissive license such as public domain or
MIT/BSD), feel free to send in a copy.

\subsection{Pulses}

If you call a function in the library that turns out to take a long time (such
as generating a 4096-bit prime), your pretty GUI will block up while the
library does something, because the event loop is not being run. Not only does
this look bad, it prevents the user from doing something else while the library
works. The way around this is to register a pulse function.

By creating a class that inherits from \type{Library\_State::UI} and
passing it to \function{global\_function}().\function{set\_ui}, you
can cause a pulse to be sent to your object occasionally (generally
when performing long running operations like prime generation;
eventually this will be extended for all operations that might take
longer than a few milliseconds).

You can recieve a pulse by implementing the member function
\function{pulse}(\type{Pulse\_Type}) within your UI class.  The
\type{Pulse\_Type} enum provides mildly useful information about the
operation in progress (for a full list of the defined
\type{Pulse\_Type} values, see \filename{enums.h}). The type code
allows you do simple feedback such as that GnuPG does during key
generation (printing various characters as the prime generation
process proceeds, such as '-' for prime test failed, '+' for prime
test worked, and so on).

Generally the thing to do inside the pulse function is to run the GUI's event
loop, for example with GTK+:

\begin{verbatim}
   while(gtk_events_pending())
      gtk_main_iteration();
\end{verbatim}

which will flush out the event queue and make your GUI seem nice and
responsive. For a particularly long-running operation (one that takes more than
a second or two), you will probably want to put up a progress bar. While you
can update it directly from the pulse function, be warned that the pulse
function is called at irregular intervals, so your progress bar's movement
might seem choppy if you update it directly from the pulse. It may be a better
move to instead set up a timer (preferably through the GUI framework) that runs
every fixed timeslice, and updates the bar when the timer goes off. As long as
the pulse function is called often enough (which is should), simply running the
event loop and letting the timer function do the updates will work fine.

\pagebreak
\section{Policy Configuration}

While Botan is performing operations on behalf on an application, there are
times where there needs to be a policy decision. For example, when generating
an X.509v3 certificate, should we include the key usage extension? Should it be
marked as a critical extension, or is non-critical OK? And so on and so
forth. It is not proper for a library to make these kinds of decisions for an
application; after all, different applications might have different needs (not
to mention the same application running at different sites). So, whenever it is
sane to do so, the library will read from an internal table to find out what it
should do when a policy decision is needed.

Right now, the option table is populated by some fixed, reasonable values at
startup. These options can then be changed by the application, either
hard-coded into the source code as an application policy, or reading them from
a file (or options screen or whatever) and setting them as the user desires
(possibly placing application-policy limits on the range they can take).

The library natively supports a simple format which is easy to parse and easy
for humans to read and write. If you're at all familiar with Windows .INI files
or OpenSSL's configs, it should be pretty easy to use. It's entirely possible
that you want to instead use an XML config (or whatever), but you'll have to
write you own parser for this (\filename{src/inifile.cpp} will provide some
ideas on what it is supposed to do).

There are basically four different things stored in the options table: strings,
numbers, booleans, and times (\emph{not} dates; times are things like ``1
hour'', ``15 minutes'', etc), though they are all represented by strings when
they are provided to the library.

\subsection{Option Types}

Strings are simply strings~--~no strings attached (sorry). A list is a
collection of strings, separated by a ':' character (no escaping is available,
so you can't actually have a ':' character in a list item).

A number (more precisely, a non-negative integer less than $2^{32}$) is
specified as a string of decimal digits~--~no special formatters (such as a
``0x'' prefix) are supported. However, you can do simply arithmetic ('+' and
'*'), and they do commute correctly. There is no explicit grouping (\ie, with
parenthesis), but generally a simple expression is all thats needed for this
sort of thing.

A boolean can take on the values true and false, which can be represented by
``true'' (and ``1'') or ``false'' (and ``0'') respectively. Unlike C, a value
of (say) ``7'' is not a boolean; it will be flagged as an error at runtime when
the library attempts to read it. Finally, a time is essentially
``\texttt{<integer>[s|m|h|d|y]}'', where integer is the magnitude and the
suffix (if present) provides a scaling value. For example ``5d'' represents 5
days, and ``60'', ``60s'', and ``1m'' all represent 60 seconds. If no suffix is
provided, the scale defaults to seconds.

\subsection{Setting and Getting Options}

The header \filename{botan/config.h} has the interface for setting
policy options. All the actual configuration options are stored in a
global object (of type \type{Config}); you can get a reference to this
object by calling \function{global\_config}.

To add (or set) an option, call
\function{global\_config}().\function{set\_option} (\type{std::string}
\arg{name}, \type{std::string} \arg{value})

To get the value of an option, there are number of member functions
which provide access, converting the underlying storage unit
(currently strings) into an appropriate base type:

\type{std::string} \function{option}(\type{std::string} \arg{option})

\type{std::vector<std::string>} \function{option\_as\_list}(\type{std::string}
\arg{option})

\type{u32bit} \function{option\_as\_u32bit}(\type{std::string} \arg{option})

\type{u32bit} \function{option\_as\_time}(\type{std::string} \arg{option})

\type{bool} \function{option\_as\_bool}(\type{std::string} \arg{option})

Simply calling \function{option} returns a \type{std::string}, which
is the underlying storage unit. If you're not sure what kind of value
might be in the type, or you want to support a type coercion that
Botan isn't supporting, you'll want to use this. Botan supports
various simple coercions, which take the underlying string as the
input. Taking the option as a list simply splits it on the ':'
character (with no escaping of any kind, eg ``abc\\:def'' splits into
``abc\\'' and ``def'')

As to defaults: strings default to the empty string, lists to an empty list,
integers default to 0, times default to no time (0 seconds), and booleans will
throw an exception if no value has been set.

\subsection{Available Options}

Generally, the defaults are chosen to provide a good level of security and
sense for typical applications. Currently, most of the options are for the
X.509 handling, since that's the place where most freedom is given to
implementations. Options are organized in a hierarchal fashion, with a
separating character of '/'. All options beginning with ``app/'' are reserved
for use by applications.

\newcommand{\confopt}[4]{
   \textbf{``#1''}, (\textbf{#2}, default \textbf{#3}): #4.
}

\begin{list}{$\cdot$}
  \item \confopt{base/memory\_chunk}{integer}{``64*1024''}{how large a
        block of memory to allocate at once}

  \item \confopt{base/default\_pbe}{string}
         {``PBE-PKCS5v20(SHA-1,TripleDES/CBC)''}{
           The default algorithm for encrypting PKCS \#8 private keys}

  \item \confopt{base/pkcs8\_tries}{integer}{3}{how many times
         \function{PKCS8::load\_key} will ask a \type{UI} object for a
         passphrase to decrypt the key before it gives up. If set to 0, it will
         continue to query the \type{UI} object until the object indicates to
         cancel the action.}

  \item \confopt{pk/blinder\_size}{integer}{64}{how long (in bits) the
        blinding factor will be when doing private-key PK operations; if set to
        zero then blinding is not performed}

  \item \confopt{pk/test/public}{string}{``basic''}{How much testing to
        perform on imported public keys; can be ``basic'' or ``all''}

  \item \confopt{pk/test/private}{string}{``basic''}{How much testing to
        perform on imported private keys; can be ``basic'' or ``all''}

  \item \confopt{pk/test/private\_gen}{string}{``all''}{How much testing to
        perform on generated private keys; can be ``basic'' or ``all''}

  \item \confopt{pem/search}{integer}{``4*1024''}{how large an area (in bytes)
        to search for PEM signatures in the heuristic that decides if data is
        PEM encoded, or raw BER data}

  \item \confopt{pem/forgive}{integer}{``8''}{how many characters that
       'look like' a PEM header will be forgiven, \ie how characters match
       before we decide it really is the PEM header, and any bad characters
       imply a malformed header}

  \item \confopt{pem/width}{integer}{``64''}{how long each PEM line will be
        encoded as; it should not be smaller than 50 or greater than 80}

  \item \confopt{rng/min\_entropy}{integer}{384}{how many bits of entropy must
        be collected before the PRNG is considered seeded}

  \item \confopt{rng/es\_files}{list}{``/dev/urandom:/dev/random''}{what paths
        to attempt reads from for entropy, typically in-kernel devices}

  \item \confopt{rng/egd\_path}{list}{``/var/run/egd-pool:/dev/egd-pool''}{
        what paths to attempt to use as an EGD socket}

  \item \confopt{rng/ms\_capi\_prov\_type}{list}{``INTEL\_SEC:RSA\_FULL''}{
        what providers the CAPI entropy source should attempt to use, in order}

  \item \confopt{rng/unix\_path}{list}{``/usr/ucb:/usr/etc:/etc''}{extra path
        fields to use when executing programs to gather entropy}

  \item \confopt{x509/validity\_slack}{time}{``24h''}{how much slack to
        allow when checking time validity on X.509 certificates}

  \item \confopt{x509/v1\_assume\_ca}{boolean}{false}{if true, then v1/v2
        X.509 certificates are considered CA certificates by default. If not
        true, then no v1/v2 certificate is considered valid for CA use}

  \item \confopt{x509/cache\_verify\_results}{time}{``30m''}{how long
        to cache certificate verification results in a \type{X509\_Store}. Set
        it to 0 if you don't want to cache the results, though this will cause
        a lot of unnecessary overhead}

  \item \confopt{x509/ca/allow\_ca}{boolean}{``false''}{whether a CA
        will allow new certificates to be marked for CA usage}

  \item \confopt{x509/ca/basic\_constraints}{string}{``always''}{can be either
        ``always'' or ``ca\_only''; if ``always'' then the basic constraints
        extension is included in new user certs as well as new CA certs}

  \item \confopt{x509/ca/default\_expire}{time}{``1y''}{how long, by
        default, a newly generated certificate is valid for}

  \item \confopt{x509/ca/signing\_offset}{time}{``30s''}{when generating a
        PKCS \#10 certificate request, it will be marked as becoming valid
        this much time before the current time; helps protect against slightly
        off clocks}

  \item \confopt{x509/ca/rsa\_hash}{string}{``SHA-1''}{what hash to use
        with an RSA key (SHA-1 is always used with DSA)}

  \item \confopt{x509/ca/str\_type}{string}{``latin1''}{what encoding to use
        by default (can be ``latin1'' or ``utf8'')}

  \item \confopt{x509/crl/unknown\_critical}{string}{``ignore''}{what
        to do when a CRL with an unknown critical extension is
        processed. Options are ``ignore'' and ``throw''. For X.509v4
        compliance, use ``ignore'', for PKIX compliance, use ``throw''}

  \item \confopt{x509/crl/next\_update}{time}{``7d''}{new CRLs are marked as
        expiring in this much time}
\end{list}

Here, in a separate list, are the options which control which extension are
included in a newly generated X.509v3 certificate, and if they should be marked
as critical extensions or not. Each one begins with ``x509/exts/'' (\ie, what is
referred to as ``basic\_constraints'' below is actually
``x509/exts/basic\_constraints''), and can take on a value of ``yes'', ``no'',
``noncritical'', or ``critical''. A value of ``no'' means the extension is not
included under any circumstances. A value of ``yes'' or ``noncritical'' (they
have the same meaning), means that the extension is included in the certificate
if there is some data to populate it with, and that the extension should be
marked as non-critical. Finally, ``critical'' means that the extension should
be marked as a critical extension. Unless otherwise noted, the option will
default to ``yes'': including the extension if data is available to fill it in,
and mark it as a non-critical extension.

A word about X.509v3 extensions: each extension can be marked either critical
or non-critical. A non-critical extension may be ignored by a compliant X.509v3
implementation (though for the common extensions, it is fairly rare for an
implementation to actually do so). On the other hand, a critical extension
forces an all-or-nothing situation: if an implementation can't handle an
extension marked critical, it is required to reject the certificate outright.

For the full meaning of the extensions, it will probably be helpful to read an
authoritative X.509 reference, such as RFC 2459 or ISO's X.509 v3/v4 documents.
The default options here were chosen to comply with the IETF PKIX X.509v3
profile, which is probably the most commonly supported X.509 profile, at least
in the United States.

\begin{list}{$\cdot$}
   \item ``basic\_constraints'' (default ``critical''): Control the use of the
         Basic Constraints extension, which marks if a certificate is a CA or
         not. Changing this is \emph{not} recommended, as this should always
         be a critical extension (doing otherwise violates most if not all
         X.509v3 profiles).
   \item
   \item ``subject\_key\_id'': Controls the use of the subject key identifier.
         Not many implementations make use of this extension, but it is not
         harmful, and it is recommended it be included in all new certificates.

   \item ``authority\_key\_id'': See comments on ``subject\_key\_id''

   \item ``subject\_alternative\_name'': Contains various pieces of information
         that don't fit into the standard certificate name, like email
         addresses and URIs. Very commonly used.

   \item ``issuer\_alternative\_name'': Like ``subject\_alternative\_name'',
         but not used nearly as often.

   \item ``key\_usage'' (default ``critical''): Marks what uses this
         certificate is valid for.

   \item ``extended\_key\_usage'': Similar to ``key\_usage'', but more general
         and much less commonly used.
\end{list}

\pagebreak
\section{Botan's Modules}

Botan comes with a variety of modules which can be compiled into the system.
These will not be available on all installations of the library, but you can
check for their availability based on whether or not certain macros are
defined.

\subsection{Pipe I/O for Unix File Descriptors}

This is a fairly minor feature, but it comes in handy sometimes. In all
installations of the library, Botan's \type{Pipe} object overloads the
\keyword{<<} and \keyword{>>} operators for C++ iostream objects, which is
usually more than sufficient for doing I/O.

However, there are cases where the iostream hierarchy does not map well to
local 'file types', so there is also the ability to do I/O directly with Unix
file descriptors. This is most useful when you want to read from or write to
something like a TCP or Unix-domain socket, or a pipe, since for simple file
access it's usually easier to just use C++'s file streams.

If \macro{BOTAN\_EXT\_PIPE\_UNIXFD\_IO} is defined, then you can use the
overloaded I/O operators with Unix file descriptors. For an example of this,
check out the \filename{hash\_fd} example, included in the Botan distribution.

\subsection{Entropy Sources}

All of these are used by the \function{Global\_RNG::seed} function if they are
available. Since this function is called by the \type{LibraryInitializer} class
when it is created, it is fairly rare that you will need to deal with any of
these classes directly. Even in the case of a long-running server that needs to
renew its entropy poll, it is easier to simply call
\function{Global\_RNG::seed} (see the section entitled ``The Global PRNG'' for
more details).

\noindent
\type{EGD\_EntropySource}: Query an EGD socket. If the macro
\macro{BOTAN\_EXT\_ENTROPY\_SRC\_EGD} is defined, it can be found in
\filename{es\_egd.h}. The constructor takes a \type{std::vector<std::string>}
that specifies the paths to look for an EGD socket.

\noindent
\type{Unix\_EntropySource}: This entropy source executes programs common on
Unix systems (such as \filename{uptime}, \filename{vmstat}, and \filename{df})
and adds it to a buffer. It's quite slow due to process overhead, and (roughly)
1 bit of real entropy is in each byte that is output. It is declared in
\filename{es\_unix.h}, if \macro{BOTAN\_EXT\_ENTROPY\_SRC\_UNIX} is
defined. If you don't have \filename{/dev/urandom} \emph{or} EGD, this is
probably the thing to use. For a long-running process on Unix, keep on object
of this type around and run fast polls ever few minutes.

\noindent
\type{FTW\_EntropySource}: Walk through a filesystem (the root to start
searching is passed as a string to the constructor), reading files. This tends
to only be useful on things like \filename{/proc} which have a great deal of
variability over time, and even then there is only a small amount of entropy
gathered: about 1 bit of entropy for every 16 bits of output (and many hundreds
of bits are read in order to get that 16 bits). It is declared in
\filename{es\_ftw.h}, if \macro{BOTAN\_EXT\_ENTROPY\_SRC\_FTW} is defined. Only
use this as a last resort. I don't really trust it, and neither should you.

\noindent
\type{Win32\_CAPI\_EntropySource}: This routines gathers entropy from a Win32
CAPI module. It takes an optional \type{std::string} which will specify what
type of CAPI provider to use. Generally the CAPI RNG is always the same
software-based PRNG, but there are a few which may use a hardware RNG. By
default it will use the first provider listed in the option
``rng/ms\_capi\_prov\_type'' which is available on the machine (currently the
providers ``RSA\_FULL'', ``INTEL\_SEC'', ``FORTEZZA'', and ``RNG'' are
recognized).

\noindent
\type{BeOS\_EntropySource}: Query system statistics using various BeOS-specific
APIs.

\noindent
\type{Pthread\_EntropySource}: Attempt to gather entropy based on jitter
between a number of threads competing for a single mutex. This entropy source
is \emph{very} slow, and highly questionable in terms of security. However, it
provides a worst-case fallback on systems which don't have Unix-like features,
but do support POSIX threads. This module is currently unavailable due to
problems on some systems.

\subsection{Compressors}

There are two compression algorithms supported by Botan, Zlib and Bzip2 (Gzip
and Zip encoding will be supported in future releases). Only lossless
compression algorithms are currently supported by Botan, because they tend to
be the most useful for cryptography. However, it is very reasonable to consider
supporting something like GSM speech encoding (which is lossy), for use in
encrypted voice applications.

You should always compress \emph{before} you encrypt, because encryption seeks
to hide the redundancy that compression is supposed to try to find and remove.

\subsubsection{Bzip2}

To test for Bzip2, check to see if \macro{BOTAN\_EXT\_COMPRESSOR\_BZIP2} is
defined. If so, you can include \filename{bzip2.h}, which will declare a pair
of \type{Filter} objects: \type{Bzip2\_Compression} and
\type{Bzip2\_Decompression}.

You should be prepared to take an exception when using the decompressing
filter, for if the input is not valid Bzip2 data, that is what you will
receive. You can specify the desired level of compression to
\type{Bzip2\_Compression}'s constructor as an integer between 1 and 9, 1
meaning worst compression, and 9 meaning the best. The default is to use 9,
since small values take the same amount of time, just use a little less memory.

The Bzip2 module was contributed by Peter J. Jones.

\subsubsection{Zlib}

Zlib compression works pretty much like Bzip2 compression. The only differences
in this case are that the macro is \macro{BOTAN\_EXT\_COMPRESSOR\_ZLIB}, the
header you need to include is called \filename{botan/zlib.h} (remember that you
shouldn't just \verb|#include <zlib.h>|, or you'll get the regular zlib API,
which is not what you want). The Botan classes for Zlib
compression/decompression are called \type{Zlib\_Compression} and
\type{Zlib\_Decompression}.

Like Bzip2, a \type{Zlib\_Decompression} object will throw an exception if
invalid (in the sense of not being in the Zlib format) data is passed into it.

In the case of zlib's algorithm, a worse compression level will be faster than
a very high compression ratio. For this reason, the Zlib compressor will
default to using a compression level of 6. This tends to give a good trade off
in terms of time spent to compression achieved. There are several factors you
need to consider in order to decide if you should use a higher compression
level:

\begin{list}{$\cdot$}
  \item Better security: the less redundancy in the source text, the harder it
        is to attack your ciphertext. This is not too much of a concern,
        because with decent algorithms using sufficiently long keys, it doesn't
        really matter \emph{that} much (but it certainly can't hurt).
  \item

  \item Decreasing returns. Some simple experiments by the author showed
        minimal decreases in the size between level 6 and level 9 compression
        with large (1 to 3 megabyte) files. There was some difference, but it
        wasn't that much.

  \item CPU time. Level 9 zlib compression is often two to four times as slow
        as level 6 compression. This can make a substantial difference in the
        overall runtime of a program.
\end{list}

While the zlib compression library uses the same compression algorithm as the
gzip and zip programs, the format is different. The zlib format is defined in
RFC 1950.

\subsubsection{Data Sources}

A \type{DataSource} is a simple abstraction for a thing that stores bytes. This
type is used fairly heavily in the areas of the API related to ASN.1
encoding/decoding. The following types are \type{DataSource}s: \type{Pipe},
\type{SecureQueue}, and a couple of special purpose ones:
\type{DataSource\_Memory} and \type{DataSource\_Stream}.

You can create a \type{DataSource\_Memory} with an array of bytes and a length
field. The object will make a copy of the data, so you don't have to worry
about keeping that memory allocated. This is mostly for internal use, but if it
comes in handy, feel free to use it.

A \type{DataSource\_Stream} is probably more useful than the memory based
one. Its constructors take either a \type{std::istream} or a
\type{std::string}. If it's a stream, the data source will use the
\type{istream} to satisfy read requests (this is particularly useful to use
with \type{std::cin}). If the string version is used, it will attempt to open
up a file with that name and read from it.

\subsubsection{Data Sinks}

A \type{DataSink} (in \filename{data\_snk.h}) is a \type{Filter} which takes
arbitrary amounts of input, and produces no output. Generally, this means it's
doing something with the data outside the realm of what
\type{Filter}/\type{Pipe} can handle, for example, writing it to a file (which
is what the \type{DataSink\_Stream} does). There is no need for
\type{DataSink}s which write to a \type{std::string} or memory buffer, because
\type{Pipe} can handle that by itself.

Here's a quick example of using a \type{DataSink}, which encrypts
\filename{in.txt} and sends the output to \filename{out.txt}. There is
no explicit output operation; the writing of \filename{out.txt} is
implicit.

\begin{verbatim}
   DataSource_Stream in("in.txt");
   Pipe pipe(new CBC_Encryption("Blowfish", "PKCS7", key, iv),
             new DataSink_Stream("out.txt"));
   pipe.process_msg(in);
\end{verbatim}

A real advantage of this is that even if ``in.txt'' is large, only as
much memory is needed for internal I/O buffers will actually be used.

\subsection{Writing Modules}

It's a lot simpler to write modules for Botan that it is to write code
in the core library, for several reasons. First, a module can rely on
external libraries and services beyond the base ISO C++ libraries, and
also machine dependent features. Also, the code can be added at
configuration time on the user's end with very little effort (\ie the
code can be distributed separately, and included by the user without
needing to patch any existing source files).

Each module lives in a subdirectory of the \filename{modules}
directory, which exists at the top-level of the Botan source tree. The
``short name'' of the module is the same as the name of this
directory. The only required file in this directory is
\filename{modinfo.txt}, which contains directives that specify what a
particular module does, what systems it runs on, and so on. Comments
in \filename{modinfo.txt} start with a \verb|#| character and continue
to end of line.

Recognized directives include:

\newcommand{\directive}[2]{
   \vskip 4pt
   \noindent
   \texttt{#1}: #2
}

\directive{realname <name>}{Specify that the 'real world' name of this module
   is \texttt{<name>}.}

\directive{note <note>}{Add a note that will be seen by the end-user at
configure time if the module is included into the library.}

\directive{require\_version <version>}{Require at configure time that
the version of Botan in use be at least \texttt{<version>}.}

\directive{define <macro>[,<macro>[,...]]}{Cause the macro
   \macro{BOTAN\_EXT\_<macro>} (for each instance of \macro{<macro>}
   in the directive) to be defined in \filename{build.h}. This should
   only be used if the module creates user-visible changes. There is a
   set of conventions that should be followed in deciding what to call
   this macro (where xxx denotes some descriptive and distinguishing
   characteristic of the thing implemented, such as
   \macro{ALLOC\_MLOCK} or \macro{MUTEX\_PTHREAD}):

\begin{itemize}
\item Allocator: \macro{ALLOC\_xxx}
\item Compressors: \macro{COMPRESSOR\_xxx}
\item EntropySource: \macro{ENTROPY\_SRC\_xxx}
\item Engines: \macro{ENGINE\_xxx}
\item Mutex: \macro{MUTEX\_xxx}
\item Timer: \macro{TIMER\_xxx}
\end{itemize}
}

\directive{<libs> / </libs>}{This specifies any extra libraries to be
linked in.  It is a mapping from OS to library name, for example
\texttt{linux -> rt}, which means that on Linux librt should be linked
in. You can also use ``all'' to force the library to be linked in on
all systems.}

\directive{<add> / </add>}{Tell the configuration script to add the
   files named between these two tags into the source tree. All these
   files must exist in the current module directory.}

\directive{<ignore> / </ignore>}{Tell the configuration script to
   ignore the files named in the main source tree. This is useful, for
   example, when replacing a C++ implementation with a pure assembly
   version.}

\directive{<replace> / </replace>}{Tell the configuration script to
   ignore the file given in the main source tree, and instead use the
   one in the module's directory.}

Additionally, the module file can contain blocks, delimited by the
following pairs:

\texttt{<os> / </os>}, \texttt{<arch> / </arch>}, \texttt{<cc> / </cc>}

\noindent
For example, putting ``alpha'' and ``ia64'' in a \texttt{<arch>} block will
make the configuration script only allow the module to be compiled on those
architectures. Not having a block means any value is acceptable.

\pagebreak
\section{Miscellaneous}

This section has documentation for anything that just didn't fit into any of
the major categories. Many of them (Timers, Allocators) will rarely be used in
actual application code, but others, like the S2K algorithms, have a wide
degree of applicability.

\subsection{S2K Algorithms}

There are various procedures (usually fairly ad-hoc) for turning a passphrase
into a (mostly) arbitrary length key for a symmetric cipher. A general
interface for such algorithms is presented in \filename{s2k.h}. The main
function is \function{derive\_key}, which takes a passphrase, and the desired
length of the output key, and returns a key of that length, deterministically
produced from the passphrase. If an algorithm can't produce a key of that size,
it will throw an exception (most notably, PKCS \#5's PBKDF1 can only produce
strings between 1 and $n$ bytes, where $n$ is the output size of the underlying
hash function).

Most such algorithms allow the use of a ``salt'', which provides some extra
randomness and helps against dictionary attacks on the passphrase. Simply call
\function{change\_salt} (there are variations of it for most of the ways you
might wish to specify a salt, check the header for details) with a block of
random data. You can also have the class generate a new salt for you with
\function{new\_random\_salt}; the salt that was generated can be retrieved with
\function{current\_salt}.

Additionally some algorithms allow you to set some sort of iteration count,
which will make the algorithm take longer to compute the final key (reducing
the speed of brute-force attacks of various kinds). This can be changed with
the \function{set\_iterations} function. Most standards recommend an iteration
count of at least 1000. ``PBKDF2(SHA-1)'', with an 8-byte salt and an iteration
count of 2048, is recommend for new applications. Currently defined S2K
algorithms are ``PBKDF1(digest)'', ``PBKDF2(digest)'', and
``OpenPGP-S2K(digest)''; you can retrieve any of these using the
\function{get\_s2k}, found in \filename{lookup.h}. As of this writing,
``PBKDF2(SHA-1)'' with 10000 iterations and an 8 byte salt is recommend for new
applications.

\subsubsection{OpenPGP S2K}

There are some oddities about OpenPGP's S2K algorithms which are documented
here. For one thing, it uses the iteration count in a strange manner; instead
of specifying how many times to iterate the hash, it tells how many
\emph{bytes} should be hashed in total (including the salt). So the exact
iteration count will depend on the size of the salt (which is fixed at 8 bytes
by the OpenPGP standard, though the implementation will allow any salt size)
and the size of the passphrase.

To get what OpenPGP calls ``Simple S2K'', set iterations to 0 (the default for
OpenPGP S2K), and do not specify a salt. To get ``Salted S2K'', again leave the
iteration count at 0, but give an 8-byte salt. ``Salted and Iterated S2K''
requires an 8-byte salt and some iteration count (this should be significantly
larger than the size of the longest passphrase that might reasonably be used;
somewhere from 1024 to 65536 would probably be about right). Using both a
reasonably sized salt and a large iteration count is highly recommended to
prevent password guessing attempts.

\subsection{Checksums}

Checksums are very similar to hash functions, and in fact share the same
interface. But there are some significant differences, the major ones being
that the output size is very small (usually in the range of 2 to 4 bytes), and
is not cryptographically secure. But for their intended purpose (error
checking), they perform very well. Some examples of checksums included in Botan
are the Adler32 and CRC32 checksums.

\subsection{Exceptions}

Sooner or later, something is going to go wrong. Botan's behavior when
something unusual occurs, like most C++ software, is to throw an exception.
Exceptions in Botan are derived from the \type{Exception} class. You can see
most of the major varieties of exceptions used in Botan by looking at
\filename{exceptn.h}. The only function you really need to concern yourself
with is \type{const char*} \function{what()}. This will return an error message
relevant to the error that occurred. For example:

\begin{verbatim}
try {
   // various Botan operations
   }
catch(Botan::Exception& e)
   {
   cout << "Botan exception caught: " << e.what() << endl;
   // error handling, or just abort
   }
\end{verbatim}

Botan's exceptions are derived from \type{std::exception}, so you don't need
to explicitly check for Botan exceptions if you're already catching the ISO
standard ones.

\subsection{Threads and Mutexes}

Botan includes a mutex system, which is used internally to lock some shared
data structures which must be kept shared for efficiency reasons (mostly, these
are in the allocation systems~--~handing out 1000 separate allocators hurts
performance and makes caching memory blocks useless). This system is supported
by the \texttt{mux\_pthr} module, implementing the \type{Mutex} interface for
systems that have POSIX threads.

If your application is using threads, you \emph{must} add the option
``thread\_safe'' to the options string when you create the
\type{LibraryInitializer} object. If you specify this option and no mutex type
is available, an exception is thrown, since otherwise you would probably be
facing a nasty crash.

There are a few functions that shouldn't be called from threads. If you want to
use them, you'll have to either do locking in your own code, or only call them
from a single thread (presumably the main thread, which initialized the
library, but that isn't required). It is assumed that most of them are called
at most once, and then the application runs. Thread-unsafe functions in Botan
include:

\begin{verbatim}
   add_engine(Engine*)
   startup_engines()
   shutdown_engines()
   set_mutex_type(Mutex*)
   set_timer_type(Timer*)
   setup_global_rng(RandomNumberGenerator*, RandomNumberGenerator*)
   destroy_global_rng()
\end{verbatim}

This list is \emph{not} complete. As you can see, most of them are used only at
startup/shutdown; the functions/objects you would tend to use regularly in an
application should be thread safe at the object level.

\subsection{Secure Memory}

A major concern with mixing modern multiuser OSes and cryptographic code is
that at any time the code (including secret keys) could be swapped to disk,
where it can later be read by an attacker. Botan stores almost everything (and
especially anything sensitive) in memory buffers which a) clear out their
contents when their destructors are called, and b) have easy plugins for
various memory locking functions, such as the \function{mlock}(2) call on many
Unix systems.

Two of the allocation method used (``malloc'' and ``mmap'') don't
require any extra privileges on Unix, but locking memory does. At
startup, each allocator type will attempt to allocate a few blocks
(typically totaling 128k), so if you want, you can run your
application \texttt{setuid} \texttt{root}, and then drop privileges
immediately after creating your \type{LibraryInitializer}. If you end
up using more than what's been allocated, some of your sensitive data
might end up being swappable, but that beats running as \texttt{root}
all the time. BTW, I would note that, at least on Linux, you can use a
kernel module to give your process extra privileges (such as the
ability to call \function{mlock}) without being root. For example,
check out my Capability Override LSM
(\url{http://www.randombit.net/projects/cap\_over/}), which makes this
pretty easy to do.

These classes should also be used within your own code for storing sensitive
data. They are only meant for primitive data types (int, long, etc): if you
want a container of higher level Botan objects, you can just use a
\verb|std::vector|, since these objects know how to clear themselves when they
are destroyed. You cannot, however, have a \verb|std::vector| (or any other
container) of \type{Pipe}s or \type{Filter}s, because these types have pointers
to other \type{Filter}s, and implementing copy constructors for these types
would be both hard and quite expensive (vectors of pointers to such objects is
fine, though).

These types are not described in any great detail: for more information,
consult the definitive sources~--~the header files \filename{secmem.h} and
\filename{allocate.h}.

\type{SecureBuffer} is a simple array type, whose size is specified at compile
time. It will automatically convert to a pointer of the appropriate type, and
has a number of useful functions, including \function{clear()}, and
\type{u32bit} \function{size()}, which returns the length of the array. It is a
template that takes as parameters a type, and a constant integer which is how
long the array is (for example: \verb|SecureBuffer<byte, 8> key;|).

\type{SecureVector} is a variable length array. Its size can be increased or
decreased as need be, and it has a wide variety of functions useful for copying
data into its buffer. Like \type{SecureBuffer}, it implements \function{clear}
and \function{size}.

\subsection{Allocators}

The containers described above get their memory from allocators. As a user of
the library, you can add new allocator methods at run time for containers,
including the ones used internally by the library, to use. The interface to
this is in \filename{allocate.h}. Basically how it works is that code needing
an allocator uses \function{get\_allocator}, which returns a pointer to an
allocator. This pointer should not be freed: the caller does not own the
allocator (it is shared among multiple users, and locks itself as needed). It
is possible to call \function{get\_allocator} with a specific name to request a
particular type of allocator, otherwise, a default allocator type is returned.

At start time, the only allocator known is a \type{Default\_Allocator}, which
just allocates memory using \function{malloc}, and \function{memset}s it to 0
when the memory is released. It is known by the name ``malloc''. If you ask for
another type of allocator (``locking'' and ``mmap'' are currently used), and it
is not available, some other allocator will be returned.

You can add in a new allocator type using \function{add\_allocator\_type}. This
function takes a string and a pointer to an allocator. The string gives this
allocator type a name to which it can be referred when one is requesting it
with \function{get\_allocator}. If an error occurs (such as the name being
already registered), this function returns false. It will return true if the
allocator was successfully registered. If you ask it to,
\type{LibraryInitializer} will do this for you.

Finally, you can set the default allocator type that will be returned using
the policy setting ``default\_alloc'' to the name of any previously registered
allocator.

\subsection{Timers}

Botan includes a pair of functions, \function{system\_time} and
\function{system\_clock}, which are used extensively in some areas of the code
(especially in the random number generators). These functions by default use
\function{std::time} and \function{std::clock}, but often you can do better
with system-dependent functions, especially for the system clock (for example,
returning the microseconds value from \function{gettimeofday}, or the
nanoseconds value from the POSIX.1b \function{clock\_gettime}, is far
superior). Modules for this exist for several systems.

You can register a new timer method with \function{set\_timer\_type}. For
example, if the \texttt{timer\_unix} module is available, one could call
\function{set\_timer\_type}(new \type{Unix\_Timer}), in which case
\function{system\_clock} will return a more ``interesting'' value based on the
return of the \function{gettimeofday} function call. This is done automatically
by the \type{LibraryInitializer} object.

\subsection{BigInt}

\type{BigInt} is Botan's implementation of a multiple-precision
integer. Thanks to C++'s operator overloading features, using \type{BigInt} is
often quite similar to using a native integer type. The number of functions
related to \type{BigInt} is quite large. You can find most of them in
\filename{bigint.h} and \filename{numthry.h}.

Due to the sheer number of functions involved, only a few, which a regular user
of the library might have to deal with, are mentioned here. Fully documenting
the MPI library would take a significant while, so if you need to use it now,
the best way to learn is to look at the headers.

Probably the most important are the encoding/decoding functions, which
transform the normal representation of a \type{BigInt} into some other form,
such as a decimal string. The most useful of these functions are

\type{SecureVector<byte>} \function{BigInt::encode}(\type{BigInt},
\type{Encoding})

\noindent
and

\type{BigInt} \function{BigInt::decode}(\type{SecureVector<byte>},
\type{Encoding})

\type{Encoding} is an enum which has values \type{Binary}, \type{Octal},
\type{Decimal}, and \type{Hexadecimal}. The parameter will default to
\type{Binary}. These functions are static member functions, so they would be
called like this:

\begin{verbatim}
  BigInt n1; // some number
  SecureVector<byte> n1_encoded = BigInt::encode(n1);
  BigInt n2 = BigInt::decode(n1_encoded);
  // now n1 == n2
\end{verbatim}

There are also C++-style I/O operators defined for use with \type{BigInt}. The
input operator understands negative numbers, hexadecimal numbers (marked with a
leading ``0x''), and octal numbers (marked with a leading '0'). The '-' must
come before the ``0x'' or '0' marker. The output operator will never adorn the
output; for example, when printing a hexadecimal number, there will not be a
leading ``0x'' (though a leading '-' will be printed if the number is
negative). If you want such things, you'll have to do them yourself.

\type{BigInt} has constructors that can create a \type{BigInt} from an unsigned
integer or a string. You can also decode a \type{byte}[] / length pair into a
BigInt. There are several other \type{BigInt} constructors, which I would
seriously recommend you avoid, as they are only intended for use internally by
the library, and may arbitrarily change, or be removed, in a future release.

An essentially random sampling of \type{BigInt} related functions:

\type{u32bit} \function{BigInt::bytes}(): Return the size of this \type{BigInt}
in bytes.

\type{BigInt} \function{random\_prime(\type{u32bit} \arg{b})}: Return a prime
number \arg{b} bits long.

\type{BigInt} \function{gcd}(\type{BigInt} \arg{x}, \type{BigInt} \arg{y}):
Returns the greatest common divisor of \arg{x} and \arg{y}. Uses the binary
GCD algorithm.

\type{bool} \function{is\_prime}(\type{BigInt} \arg{x}): Returns true if
\arg{x} is a (possible) prime number. Uses the Miller-Rabin probabilistic
primality test with fixed bases. For higher assurance, use
\function{verify\_prime}, which uses more rounds and randomized 48-bit bases.

\subsubsection{Efficiency Hints}

If you can, always use expressions of the form \verb|a += b| over
\verb|a = a + b|. The difference can be \emph{very} substantial, because the
first form prevents at least one needless memory allocation, and possibly as
many as three.

If you're doing repeated modular exponentiations with the same modulus, create
a \type{BarrettReducer} ahead of time. If the exponent or base is a constant,
use the classes in \filename{mod\_exp.h}. This stuff is all handled for you by
the normal high-level interfaces, of course.

Never use the low-level MPI functions (those that begin with
\texttt{bigint\_}). These are completely internal to the library, and
may make arbitrarily strange and undocumented assumptions about their
inputs, and don't check to see if they are actually true, on the
assumption that only the library itself calls them, and that the
library knows what the assumptions are. The interfaces for these
functions can change completely without notice.

\pagebreak
\section{Algorithms}

\subsection{Recommended Algorithms}

This section is by no means the last word on selecting which algorithms to use.
However, Botan includes a sometimes bewildering array of possible algorithms,
and unless you're familiar with the latest developments in the field, it can be
hard to know what is secure and what is not. The following attributes of the
algorithms were evaluated when making this list: security, standardization,
patent status, support by other implementations, and efficiency (in roughly
that order).

It is intended as a set of simple guidelines for developers, and nothing more.
It's entirely possible that there are algorithms in Botan that will turn out to
be more secure than the ones listed, but the algorithms listed here are
(currently) thought to be safe.

\begin{list}{$\cdot$}
  \item Block ciphers: TripleDES or AES in CBC mode with ``PKCS7'' padding.
  \item

  \item Stream Ciphers: Use any of the recommended block ciphers in CTR mode.

  \item Hash functions: SHA-1, SHA-256, SHA-512

  \item MACs: HMAC with any recommended hash function

  \item Public Key Encryption: RSA with ``EME1(SHA-1)''

  \item Public Key Signatures: RSA with EMSA4 and any recommended hash, or DSA
     with ``EMSA1(SHA-1)''

  \item Key Agreement: Diffie-Hellman, with ``KDF2(SHA-1)''
\end{list}

\subsection{Compliance with Standards}

Botan is/should be compatible with many cryptographic standards, including the
following:

\newcommand{\standard}[2]{
   \vskip 4pt
   * #1: \textbf{#2}
}

\standard{RSA}{PKCS \#1 v2.1, ANSI X9.31}

\standard{DSA}{ANSI X9.30, FIPS 186-2}

\standard{Diffie-Hellman}{ANSI X9.42, PKCS \#3}

\standard{Certificates}{ITU X.509, RFC 3280/3281 (PKIX), PKCS \#9 v2.0,
PKCS \#10}

\standard{Private Key Formats}{PKCS \#5 v2.0, PKCS \#8}

\standard{DES/DES-EDE}{FIPS 46-3, ANSI X3.92, ANSI X3.106}

\standard{SHA-1}{FIPS 180-2}

\standard{HMAC}{ANSI X9.71, FIPS 198}

\standard{ANSI X9.19 MAC}{ANSI X9.9, ANSI X9.19}

\vskip 8pt
\noindent
There is also support for the very general standards of \textbf{IEEE 1363-2000}
and \textbf{1363a}. Most of the contents of such are included in the standards
mentioned above, in various forms (usually with extra restrictions which 1363
does not impose).

\subsection{Algorithms Listing}

Botan includes a very sizable number of cryptographic algorithms. In
nearly all cases, you never need to know the header file or type name
to use them. However, you do need to know what string (or strings) are
used to identify that algorithm. Generally, these names conform to
those set out by SCAN (Standard Cryptographic Algorithm Naming), which
is a document which specifies how strings are mapped onto algorithm
objects, which is useful for a wide variety of crypto APIs (SCAN is
oriented towards Java, but Botan and several other non-Java libraries
also make at least some use of it). For full details, read the SCAN
document, which can be found at
\url{http://www.users.zetnet.co.uk/hopwood/crypto/scan/}

Many of these algorithms can take options (such as the number of
rounds in a block cipher, the output size of a hash function,
etc). These are shown in the following list; all of them default to
reasonable values (unless otherwise marked). There are
algorithm-specific limits on most of them. When you see something like
``HASH'' or ``BLOCK'', that means you should insert the name of some
algorithm of that type. There are no defaults for those options.

A few very obscure algorithms are skipped; if you need one of them,
you'll know it, and you can look in the appropriate header to see what
that classes' \function{name} function returns (the names tend to
match that in SCAN, if it's defined there).

\begin{list}{$\cdot$}
  \item ROUNDS: The number of rounds in a block cipher.
  \item
  \item OUTSZ: The output size of a hash function or MAC
  \item PASS: The number of passes in a hash function (more passes generally
               means more security).
\end{list}

\vskip .05in
\noindent
\textbf{Block Ciphers:} ``AES'', ``Blowfish'', ``CAST-128'',
``CAST-256'', ``DES'', ``DESX'', ``TripleDES'', ``GOST'', ``IDEA'',
``MARS'', ``MISTY1(ROUNDS)'', ``RC2'', ``RC5(ROUNDS)'', ``RC6'',
``SAFER-SK(ROUNDS)'', ``SEED'', ``Serpent'', ``Skipjack'', ``Square'',
``TEA'', ``Twofish'', ``XTEA''

\noindent
\textbf{Stream Ciphers:} ``ARC4'', ``MARK4'', ``Turing'', ``WiderWake4+1-BE''

\noindent
\textbf{Hash Functions:} ``FORK-256'', ``HAS-160'', ``MD2'', ``MD4'', ``MD5'',
``RIPEMD-128'', ``RIPEMD-160'', ``SHA-160'', ``SHA-256'', ``SHA-384'',
``SHA-512'', ``Tiger(OUTSZ,PASS)'', ``Whirlpool''

\noindent
\textbf{MACs:} ``HMAC(HASH)'', ``CMAC(BLOCK)'', ``X9.19-MAC''

\subsection{Removing Algorithms}

You may well want to remove some of Botan's algorithms in order to fit it into
a memory-constrained system, where you're counting the kilobytes. For the most
part, this is trivial to do, and Botan's interface makes it easy for
applications to test for the presence of an algorithm at runtime, so a
well-behaved application can work without any need for porting on such an
version of Botan.

In some versions of 1.3.x, you can use the 'minimal' module, which removes
large amount of Botan, including most ciphers and hashes (except AES, DES/3DES,
SHA-1, HMAC, RSA, DSA, and Diffie-Hellman), DLIES, EAX and CTS modes, and a few
other odds and ends. You can check for this being the case by seeing if
\macro{BOTAN\_EXT\_MINIMAL} is defined, though for the most part it's better to
use the lookup interface (since you have no way of knowing what exactly the
minimal module might remove from release to release, and certainly not if the
shared object you're linking to has a particular algorithm). This module was
removed just before 1.4.0, as there is a better way to handle all of this in
the new engine code, which is aware of things outside public key algorithms.

Removing things like the PK signature encoding schemes (EMSA2, EMSA3...) is
somewhat more complicated and not documented here (thought it is actually quite
simple if you know how to do it -- the minimal module shows how). This tutorial
(of sorts) will go through the steps required to compile a version of Botan
without the Blowfish block cipher (which has been included since the first
release of Botan, in the spring of 2001).

The first step is to remove the files \filename{include/blowfish.h},
\filename{src/blowfish.cpp}, and \filename{src/blfs\_tab.cpp}, which actually
implement the algorithm. Then minor editing of \filename{src/algolist.cpp} is
required. First, remove the line that includes the Blowfish header
\filename{botan/blowfish.h}. Then look in \function{get\_block\_cipher} for the
code that adds a Blowfish block cipher object to the internal lookup table, and
remove it. Run the configure script, and then \textbf{make} the library. Tada!
Done.

So how does an application test for such a situation? The first is to simply
try to pass the name ``Blowfish'' to constructor of \type{CBC\_Encryption} or
other Botan \type{Filter}, and catch the resulting exception. This is not
particularly flexible, though. If an application wants to check on the status
of Botan's support for a particular algorithm, it can call some status
functions found in \filename{lookup.h}, called \function{have\_block\_cipher},
\function{have\_stream\_cipher}, \function{have\_hash}, and
\function{have\_mac}, passing in the name of the desired algorithm. If Botan
knows about it, the function will return true.

There are a handful of algorithms which are considered ``sacred'', in that an
application can always expect that they exist, and a distributor or other
end-user should not remove them without considering the possibly serious
consequences. At this time, these are: AES, DES, TripleDES, SHA-1, and HMAC.
This allows a workable fallback strategy for applications.

One other useful application of this is to remove patented algorithms, for
example if Botan were to be included as part of a commercial Linux
distribution.

For the most part, applications don't have to really worry about this, simply
because the cases this will be required are fairly rare. Checking for the
availability of patented algorithms like RC5 and IDEA before using them might
be a good idea, though.

Another advantage of this is that an application can be written to take
advantage of an algorithm which is not currently part of Botan. If it's not
available, one can simply fall back on another algorithm, and when/if it is
added to Botan, the application will start using it automagically.

\subsection{Compatibility}

Generally, cryptographic algorithms are well standardized, and thus
compatibility between implementations is relatively simple (of course, not all
algorithms are supported by all implementations). But there are a few
algorithms which are poorly specified, and these should be avoided if you wish
your data to be processed in the same way by another implementation (including
future versions of Botan).

The block cipher GOST has a particularly poor specification: there are no
standard Sboxes, and the specification does not give test vectors even for
sample boxes, which leads to issues of endian conventions, etc.

If you wish maximum portability between different implementations of an
algorithm, it's best to stick to strongly defined and well standardized
algorithms, TripleDES, AES, HMAC, and SHA-1 all being good examples.

\pagebreak
\section{Support and Further Information}

\subsection{Patents}

Some of the algorithms implemented by Botan may be covered by patents in some
locations. Algorithms known to have patent claims on them in the United States
and which are not available in a license-free/royalty-free manner include:
IDEA, MISTY1, RC5, RC6, and Nyberg-Rueppel.

You must not assume that, just because an algorithm is not listed here, it is
not encumbered by patents. If you have any concerns about the patent status of
any algorithm you are considering using in an application, please discuss it
with your attorney.

\subsection{Recommended Reading}

It's a very good idea if you have some knowledge of cryptography prior
to trying to use this stuff. You really should read one or more of
these books before seriously using the library (note that the Handbook
of Applied Cryptography is available for free online):

\setlength{\parskip}{5pt}

\noindent
\textit{Handbook of Applied Cryptography}, Alfred J. Menezes,
Paul C. Van Oorschot, and Scott A. Vanstone; CRC Press

\noindent
\textit{Security Engineering -- A Guide to Building Dependable Distributed
Systems}, Ross Anderson; Wiley

\noindent
\textit{Cryptography: Theory and Practice}, Douglas R. Stinson; CRC Press

\noindent
\textit{Applied Cryptography, 2nd Ed.}, Bruce Schneier; Wiley

\noindent
Once you've got the basics down, these are good things to at least take a look
at: IEEE 1363 and 1363a, SCAN, NESSIE, PKCS \#1 v2.1, the security related FIPS
documents, and the CFRG RFCs.

\subsection{Support}

Questions or problems you have with Botan can be directed to the
development mailing list. Joining this list is highly recommended if
you're going to be using Botan, since often advance notice of upcoming
changes is sent there. ``Philosophical'' bug reports, announcements of
programs using Botan, and basically anything else having to do with
Botan are also welcome.

The lists can be found at
\url{http://www.randombit.net/mailman/listinfo/}.

\subsection{Contact Information}

A PGP key with a fingerprint of
\verb|621D AF64 11E1 851C 4CF9 A2E1 6211 EBF1 EFBA DFBC| is used to sign all
Botan releases. This key can be found in the file \filename{doc/pgpkeys.asc};
PGP keys for the developers are also stored there.

\vskip 5pt \noindent
Web Site: \url{http://botan.randombit.net}

\vskip 5pt \noindent
Mailing lists: \url{http://www.randombit.net/mailman/}

\subsection{License}

Copyright \copyright  2000-2006, The Botan Project

This work is licensed under the Creative Commons
Attribution-ShareAlike 2.5 License. To view a copy of this license,
visit \url{http://creativecommons.org/licenses/by-sa/2.5/} or send a
letter to Creative Commons, 543 Howard Street, 5th Floor, San
Francisco, California, 94105, USA.

\end{document}
